% Copyright (C) 2020  George P. W. Williams, Jr.
\chapter{Details of GECO Classes and Functionality}

This chapter provides a more detailed discussion of each of \geco's 
classes, and the functionality implemented by their methods.   This functionality 
includes both the state retained by instances of each class (their {\em slots}),
and the functions (both generic and otherwise) which operate on those instances.

Even with all the functionality which \geco\ implements, it will still be
necessary to define some things which are specific to your application.
Generally this will be done by specializing \geco's classes (\ie, defining
some subclasses of \geco's builtin classes), and adding a few method
definitions to override and/or extend some of \geco's default behaviors.

Terminology Notes:

\begin{itemize}

\item In the material which follows, a statement which refers to `an instance of
\bold{a} {\it class-name} class' means that the instance is of the class
{\it class-name} or one of its subclasses. If the intent is to restrict the
instance to being of the named class, excluding subclasses, the wording will be
of the form `an instance of \bold{the} {\it class-name} class.'

\item In the descriptions of the methods, it will often be necessary to
distinguish between a generic function, a method for the generic function, and
the specific method supplied by \geco. A generic function and a method (as
specialized in the \term{flag line} above the description) are both parts of a
functional protocol which \geco\ expects to be honored.  A description
of a \geco-supplied method specifies that it
implements (fulfills) the requirements of this protocol.  \Geco\ may
define multiple methods (specialized for different classes) to implement
the generic function protocol for different classes.

\end{itemize}

For each class, the following sections will present the slots which 
are present in instances of the class (\ie, the values stored with each 
instance) and the functionality which has been defined for use with 
instances of the class (and its subclasses).
Generally, GAs implemented with \geco\ will not instantiate these classes.  
Instead, it will be more common to define subclasses which extend these 
classes (via added slots and methods) and specialize them (by overriding 
and/or extending inherited methods).


\section{The Ecosystem Class}

An \inxclass{ecosystem} is the highest level abstraction in a \geco\ implementation.
It is also the handle for manipulating a particular run of a GA. Since there may be
more than one instance of an \inxclass{ecosystem} in existence at one time, it is
possible to use \geco\ to create applications which use more than one GA at the same
time. The individual GAs could be competing, working on separate aspects of the same
problem, or they could be completely independent.
\filbreak
{\samepage

\Defclass {ecosystem}

\gap
\bold{Instance Allocated Slots}

  \Defslot {population}
  \defaccessor {population}

  An instance of a \inxclass{population} class.
  The population of an ecosystem is the set of organisms which are being 
  evolved.
\par}%end samepage

\filbreak

{\samepage
  \Defslotv {generation-number} {0}
  \defaccessor {generation-number}

  An integer, initially 0, which is 
  incremented each time the population enters a new {\em generation}.
  This happens each time the \inxgeneric{evolve}
  function is invoked on an \inxclass{ecosystem} instance (including
  \inxgeneric{evolve}'s recursive self-invocations). It should be noted
  that the actual creation of new members of the population (or creation
  of an entire new population) is expected to be handled by a \inxgeneric{regenerate}
  method specialized on a \inxclass{population} class, which
  is called by the default \geco-supplied \inxmethod{regenerate} method 
  specialized on \inxclass{ecosystem}.
\par}%end samepage

\filbreak

{\samepage
  \Defslotv {evaluation-number} {0}
  \defaccessor {evaluation-number}

  An integer, initially 0, which counts the number of times the 
  \inxgeneric{evaluate} function is applied to an \inxclass{organism} instance.
\par}%end samepage

\filbreak

{\samepage
  \Defslot {plan}
  \defaccessor {plan}

  An instance of a \inxclass{genetic-plan} class.
\par}%end samepage
\gap

\filbreak

The number of generations and evaluations are tracked by \geco\ so that the 
GA can be terminated based on the number of generations or evaluations 
exceeding some specific maximum limits, specified by the GA implementor.  
These limits are among the slots of the class \inxclass{genetic-plan}.

\filbreak
The \inxslot{population} and \inxslot{plan} are distinguished from the 
\inxclass{ecosystem} so that their classes may be specialized independently.  
Thus an instance of a single \inxclass{population} class may be manipulated 
using different \inxclass{plan}s, while instances of a single \inxclass{plan} may be 
used with different \inxclass{population}s.

\gap

\filbreak

{\samepage
	\bold{Instance Creation and Initialization}

The \inxclass{ecosystem} instance initialization has been extended by \geco\ by providing
a \cl{shared-initialize :AFTER} method for \inxclass{ecosystem}
to process the keyword arguments described below as follows:
\begin{itemize}
	\item If the call to \inxfun{make-instance}
	to create an instance of \inxclass{ecosystem} does not supply an initial population,
	then the \cl{:AFTER} method will initialize the \inxslot{population} slot using
	\inxgeneric{make-population}, with the values supplied via the \arg{:pop-class} and
	\arg{:pop-size} initargs.
	The call to \inxgeneric{make-population} also passes a value of \cl{t} for the \cl{:random}
	keyword argument, causing the initial population to be initialized to random organisms.
	(See Section~\ref{sec:population},
	page~\pageref{sec:population}.)
	
	\item Similarly, if a \inxclass{genetic-plan} instance is not supplied, then the \cl{:AFTER}
	method will initialize the \inxslot{plan} slot using \inxgeneric{make-genetic-plan}, with
	the value supplied via the \arg{:plan-class} initarg, and initializes the plan
	instance's \inxslot{generation-limit} and \inxslot{evaluation-limit} slots using the
	\arg{:gneration-limit} and \arg{:evaluation-limit} initargs. (See Section~\ref{sec:genetic-plan},
	page~\pageref{sec:genetic-plan}.)
\end{itemize}
The \inxgeneric{make-population} and \inxgeneric{make-genetic-plan} generic functions and
methods will be described subsequently, along with other \inxclass{ecosystem}-specialized
methods.
No special functions for the creation of \inxclass{ecosystem} instances have been defined,
since \inxgeneric{make-instance} and the standard \term{CLOS} protocol it follows provide all
the necessary functionality.

\par}%end samepage

\filbreak

{\samepage
\Definitarg {:plan-class}

  Provide the class for the \inxclass{genetic-plan} to be used by the
\inxclass{ecosystem}.
\par}%end samepage

\filbreak

{\samepage
\Definitarg {:pop-class}

  Provide the class for the \inxclass{population} instances to be created by
the \inxclass{ecosystem}.
\par}%end samepage

\filbreak

{\samepage
\Definitarg {:pop-size}

  Specifies the size to be used when the \inxclass{ecosystem} creates
\inxclass{population} instances.
\par}%end samepage

\filbreak

{\samepage
\Definitarg {:generation-limit}

This initarg is intended to specify the maximum number of generations which the ecosystem
will be allowed to evolve.
This limit is typically enforced by the \inxgeneric{evolution-termination-p} function
(see page~\pageref{evolution-termination-p}).
	
\par}%end samepage

\filbreak

{\samepage
\Definitarg {:evaluation-limit}

This initarg is intended to specify the maximum number of evaluations which the ecosystem
will be allowed to perform.
This limit is typically enforced by the \inxgeneric{evolution-termination-p} function
(see page~\pageref{evolution-termination-p}).

\par}%end samepage

\filbreak
\gap

{\samepage
\bold{Specialized Methods}

No special functions for the creation of \inxclass{ecosystem} instances have been defined
in \geco, since the \inxgeneric{make-instance} function and the standard \term{CLOS} protocol
it follows provide all the necessary functionality.
\par}% end \samepage

\filbreak

{\samepage

% use \Eggeneric so the CLOS generic isn't indexed
\Eggeneric shared-initialize {standard-object slot-names \rest initargs}
\defaftermethod shared-initialize {(ecosystem \inxclass{ecosystem}) slot-names \rest initargs
	\key :plan-class :pop-class :pop-size :generation-limit :evaluation-limit}

This method extends the initialization for \inxclass{ecosystem} instances to provide for the 
automatic creation and initialization of the \inxclass{population} and \inxclass{genetic-plan}
instances and slots.
The \inxgeneric{make-population} and \inxgeneric{make-genetic-plan} generic functions (described
next) are provided to support customization of these actions.
The call to \inxgeneric{make-population} in the default \geco-supplied method
passes a value of \cl{t} for the \cl{:random} keyword
argument, causing the initial population to be initialized to random organisms.
If \cl{:generation-limit} and \cl{evaluation-limit} are specified, the \inxslot{generation-limit}
and \inxslot{evaluation-limit} slots in the \inxclass{genetic-plan} instance are also initialized.
\par}% end \samepage

\filbreak

{\samepage

\Defgeneric make-population {ecosystem population-class \key :size :random}
\defmethod make-population {(ecosystem \inxclass{ecosystem}) population-class
	\key :size :random}
	\label{method:make-population}

This function provides an abstract interface to creation of a \inxclass{population} instance
to store in the \inxslot{population} slot of \arg{ecosystem}.
The primary \geco-supplied method
invokes \inxgeneric{make-instance} on the class \arg{population-class}, passing
the \arg{:size} argument, which determines the population size, and the
\arg{:random} argument, which if it is non-\cl{nil}, will cause the population to
be created with random organisms (intended for creation of the initial population).
It also returns the \inxclass{population} instance thus created.
\par}%end samepage

\filbreak

{\samepage
\Defgeneric make-genetic-plan {ecosystem genetic-plan-class}
\defmethod make-genetic-plan {(ecosystem \inxclass{ecosystem}) genetic-plan-class}
	\label{method:make-genetic-plan}

This function provides an abstract interface to creation of the \inxclass{genetic-plan}
instance for \arg{ecosystem}.  The \geco-supplied primary method invokes
\inxgeneric{make-instance} on \arg{genetic-plan-class}, and also supplies
\arg{ecosystem} so that the plan can be linked to the ecosystem (and vice-versa).
It also returns the \inxclass{genetic-plan} instance thus created.
\par}%end samepage

\filbreak

{\samepage
\Defgeneric evolve {ecosystem}
\defmethod evolve {(ecosystem \inxclass{ecosystem})}

This is the principle function which will be used by GA developers to invoke
their algorithm. The \geco-supplied primary method calls \inxmethod{evaluate} on
\arg{ecosystem}, and if the termination\index{termination} condition has not been
reached (see \inxgeneric{evolution-termination-p}), creates a new generation of
its \inxslot{population} via the \inxgeneric{regenerate} function, and
recurses to evolve some more\footnote{See the discussion in the footnote on
page~\pageref{recursive-vs-iterative-evolve}}.
The value returned by this function is not defined.
\par}%end samepage

\filbreak

{\samepage  
\Defgeneric evaluate {thing genetic-plan}
\defmethod evaluate {(ecosystem \cl{ecosystem}) genetic-plan}

The purpose of this function is to cause \arg{thing} to be evaluated according to
the specified \term{genetic plan}. The \geco-supplied primary method for
\inxclass{ecosystem} instances evaluates \arg{ecosystem} by calling
\inxgeneric{evaluate} on its \inxslot{population} with \arg{genetic-plan}. (Also
see the \inxmethod{evaluate} method specialized for the class \inxclass{population}, on
page~\pageref{evaluate-population}.)
The value returned by this function is not defined.
\par}%end samepage

\filbreak


\section{The Population Class} \label{sec:population}

A population is the most global structure upon which a GA operates. Although
\term{genetic operator}s are applied to the members (\term{organism}s in 
\geco's terminology, though they are often called
\ital{individuals}) of a population, it is at the level of the population that
the GA is really working.

\filbreak

{\samepage
\Defclass {population}

Instances of \inxclass{population} classes collect all the \inxclass{organism} 
instances of a generation.
\par}%end samepage

\gap

\filbreak

{\samepage
\bold{Instance Allocated Slots}

\Defslot {ecosystem}
\definitarg {:ecosystem}
\defaccessor {ecosystem}

Provides a link back to the \inxclass{ecosystem} instance to which the population belongs.
\par}%end samepage

\filbreak

{\samepage
\Defslot {organisms}
\defaccessor {organisms}

A vector, which contains all the \inxclass{organism} instances in the population.
\par}%end samepage

\filbreak

{\samepage
\Defslotv {size} {nil}
\definitarg {:size}
\defaccessor {size}

Either \cl{nil} or an integer, which indicates the size of the population, \ie,
the size of the vector in the \inxslot{organisms} slot. When \cl{nil}, the
organism vector will not be created automatically. \par}%end samepage

\filbreak

{\samepage
\Defslot {statistics}
\definitarg {:statistics}
\defaccessor {statistics}

An instance of a \inxclass{population-statistics} class, which holds statistics
\geco\ needs for the population. The \inxclass{population-statistics} class is
distinct from the \inxclass{population}class so that their sub-classes and methods may 
be specialized independently. 
\par}%end samepage

\gap
\filbreak

{\samepage

\bold{Instance Creation and Initialization}

The generic function \inxgeneric{make-population} (see page~\pageref{method:make-population})
is the \geco\ interface for creation of \inxclass{population} instances.

The initialization for instances of \inxclass{population} has been extended to
provide for automatic creation and initialization of the \inxslot{organisms}
vector. The functions \inxgeneric{make-organisms-vector} and
\inxgeneric{make-organisms} are used to permit customization of these
initialization actions; \inxgeneric{make-organisms-vector} is called when the
\inxslot{size} slot has a non-\cl{nil} value, and \inxgeneric{make-organisms} is
called only when both \inxslot{size} and \inxinitarg{:random} (below) have
non-\cl{nil} values. It is the responsibility of the \term{genetic plan} to
create the organisms after the initial generation.
\par}%end samepage

\filbreak

{\samepage
The \inxclass{population} instance initialization has been extended to support the following
additional initarg:

\Definitarg {:random}

The value of this keyword is passed to \inxgeneric{make-organisms}, and is intended
to support automatic initialization of the initial population to random organisms.
\par}%end samepage
\gap
  
\filbreak

{\samepage

\bold{Specialized Methods}

Note that most (if not all) of the generic functions in
Section~\ref{sec:selection-methods}, Selection Methods, have methods which
are specialized on the \inxclass{population} class.

\filbreak

{\samepage
	
% use \Eggeneric so the CLOS generic isn't indexed
\Eggeneric shared-initialize {standard-object slot-names \rest initargs}
\defaftermethod shared-initialize {(pop \inxclass{population}) slot-names \rest initargs
	\key :random}

This method extends the initialization for \inxclass{population} instances to provide for the 
automatic creation and initialization of the \inxclass{organism} instances for the population.
In the \geco-supplied default method, if the \inxslot{organisms} slot of 
\arg{pop} is unbound, the \inxgeneric{make-organisms-vector} generic function
(see below) is invoked to initialize the \inxslot{organisms}
slot; and then, if the \arg{:random} keyword argument is non-\cl{nil}, the
\inxgeneric{make-organisms} function (see below) is invoked to initialize
the population's organism vector to random organisms.
\par}% end \samepage

\filbreak

{\samepage
\Defgeneric make-organisms-vector {population size}
\defmethod make-organisms-vector {(population \inxclass{population}) size}

This function provides an abstract interface to creation of the population's
organisms vector (the vector which holds \arg{population}'s organisms). The
\arg{size} argument determines the size of the vector. The \geco-supplied primary
method uses the Common Lisp function \cl{make-array} to create an array of the
specified size.
It also returns the the array thus created.
\par}%end samepage

\filbreak

{\samepage
\Defgeneric make-organisms {population \key :random}
\defmethod make-organisms {(population \inxclass{population}) \key :random}

This function provides an abstract interface to creation of the organisms in
\arg{population}'s organisms vector. The \arg{:random} argument, when
non-\cl{nil}, causes all the new organisms to be random (\ie, have randomly
chosen chromosomes). The \geco-supplied primary method invokes
\inxgeneric{make-organism} for each position in the organisms vector. The
\arg{:random} argument is passed to each call to \inxgeneric{make-organism}.
The value returned by this function is not defined.
\par}%end samepage

\filbreak

{\samepage
\Defgeneric make-organism {population \key :random :no-chromosome}
\defmethod make-organism {(population \inxclass{population}) \key :random :no-chromosome}
	\label{method:make-organism}

This function provides an abstract interface to creation of a single organism
based on the \inxgeneric{organism-class} of \arg{population}. The \arg{:random}
argument, when non-\cl{nil}, causes the new organism to be random (\ie, have
randomly chosen chromosomes). The \arg{:no-chromosome} argument, when
non-\cl{nil}, causes the organism to be created without chromosomes, avoiding
wasted work when the chromosomes will be supplied by other mechanisms, \eg,
\term{genetic operator}s. The \geco-supplied primary method passes \arg{population} to
the call to \inxgeneric{make-instance} so that the organism can have a back-link
to the population to which it belongs. The \arg{:random} and
\arg{:no-chromosomes} arguments are passed to \inxgeneric{make-instance}.
It also returns the the \inxclass{organism} instance thus created.
\par}%end samepage

\filbreak

{\samepage
\Defgeneric organism-class {population}

This function returns the class to be used to create organisms which will become members
of \arg{population}. The GA developer {\em must implement the primary method} for all
subclasses of the class \inxclass{population}. \Geco\ does not provide a default primary method
specialized on the \inxclass{population} class.\footnote{There are comments at the beginning
of the {\tt generics.lisp} file which summarize the functions which should or must be
defined to implement a working GA using \geco.}
\par}%end samepage

\filbreak

{\samepage
\Defgeneric evaluate {thing genetic-plan}
\defmethod evaluate {(population \inxclass{population}) (genetic-plan \inxclass{genetic-plan})}
\label{evaluate-population}

This function evaluates \arg{thing} according to \arg{genetic-plan}. This method
assures that each organism in \arg{population} is evaluated. The \geco-supplied
primary method only calls \inxgeneric{evaluate} on an organism if the organism
doesn't already have a \term{score} in its \inxslot{score} slot. After
\arg{population} has been evaluated, \inxgeneric{normalize-score} and
\inxgeneric{make-population-statistics} are called to assure that normalized
scores and statistics have been computed for the population.
The value returned by this function is not defined.
\par}%end samepage

\filbreak

{\samepage
\Defgeneric make-population-statistics {population}
\defmethod make-population-statistics {(population \inxclass{population})}
	\label{method:make-population-statistics}

This function provides an abstract interface to creation of the
\inxclass{population-statistics} instance for \arg{population}, based on the
\inxgeneric{population-statistics-class} of \arg{population}. The \geco-supplied
primary method passes \arg{population} to \inxgeneric{make-instance} so that the instance can
have a back-link to the population to which it belongs.
It also returns the \inxclass{population-statistics} instance thus created.
\par}%end samepage

\filbreak

{\samepage
\Defgeneric compute-statistics {population}
\defmethod compute-statistics {(population \inxclass{population})}

This function provides an abstract interface for computing statistics for
\arg{population}. This method provieds a place for a population class to provide
for customization of statistics computation. The \geco-supplied primary method
simply calls \inxgeneric{compute-statistics} on the statistics instance of
\arg{population}. (Also see the description of \inxmethod{compute-statistics}
on page~\pageref{compute-population-statistics},
specialized on the class \inxclass{population-statistics}, and
Section~\ref{sec:pop-stats-class}, which details the statistics that are computed.)
The value returned by this function is not defined.
\par}%end samepage

\filbreak

{\samepage
\Defgeneric compute-binary-allele-statistics {population}
\defmethod compute-binary-allele-statistics {(population \inxclass{population})}

This function returns a list of vectors (one per binary chromosome in the
organisms of \arg{population}) of counts (\cl{fixnum}s), by locus, of non-zero
alleles. For example, if the organisms in a population contain $c$ binary
chromosome (and any number of non-binary chromosomes), and each binary chromosome
contains $b$ loci, then this function will return a list containing $c$ vectors
of $b$ fixnums. Each \cl{fixnum} in the returned vectors is a count of non-zero
alleles in the entire population at the locus whose index corresponds to the
index into the $c^{\rm th}$ vector of counts. \Eg, if the third count in the
first vector is 7, then the entire population contains 7 non-zero alleles in
locus 3 of the first binary chromosome of each organism.
\par}%end samepage

\filbreak

{\samepage
\Defgeneric normalize-score {thing plan}
\defmethod normalize-score {(population \inxclass{population})
                            \hbox{(genetic-plan \inxclass{genetic-plan})}}

This function computes the normalized
\term{score}(s)\index{score!normalization}\index{normalization} for \arg{thing}.
This method computes the normalized scores for all organisms in \arg{population}.
The \geco-supplied primary method for \inxclass{population} invokes
\inxgeneric{normalize-score} (see page \pageref{method:normalize-score:organism})
for each organism in \arg{population}, according to the \arg{genetic-plan}, and
updates \cl{(statistics \arg{population})} with normalized values using the function
\inxgeneric{compute-normalized-statistics}.
The value returned by this function is not defined.

[Note that \geco\ version 2.1 changed the calling sequence for this generic function and all its methods.]
\par}%end samepage

\filbreak

There are a number of different ways to normalize\index{normalization} the
scores. With some plans and evaluation functions, it may not even be necessary,
though beware that the score should always be $\ge 0$ (see Chapter 4 of
\cite{ga:goldberg}, under the sections on Scaling Mechanisms and Ranking
Procedures).

\filbreak

{\samepage
\Defgeneric population-statistics-class {population}
\defmethod population-statistics-class {(population \inxclass{population})}

This function returns the population-statistics class which will be used for
\arg{population.} The \geco-supplied primary method specialized for the
\inxclass{population} class returns \inxclass{population-statistics}.
\par}%end samepage

% This function could be performed by \cl{:allocation :per-class} slots 
% if and when they can be implemented portably in Common Lisp.

\filbreak

{\samepage
\Defgeneric converged-p {population}
\defmethod converged-p {(population \inxclass{population})}
	\label{population:converged-p}
This function is a predicate which indicates whether \arg{population} has
\concept{converged}, which is useful as a termination\index{termination}
condition. The \geco-supplied primary method defines convergence as
either of the following:
\par}%end samepage

\filbreak

{\samepage
  \begin{enumerate}
    \item All organisms in \arg{population} have the same \inxslot{score}; or
    \item At least a portion of \arg{population} (specified by the
        \inxgeneric{convergence-fraction} function) has a 
        \inxslot{normalized-score} which is {\em as good as} the value specified by the
        \inxgeneric{convergence-threshold-margin} function.
  \end{enumerate}
}%end samepage

\filbreak

Note that this allows \geco\ to either {\em maximize} or {\em minimize}
\term{scores}. The mechanism for determining whether \geco\ maximizes or minimizes,
and hence how it determines {\em as good as} or {\em better than}, is determined by
mixing one of two classes with the population class used by the GA. These
\term{mixin classes} are described below, in section
\ref{sec:population-mixin-classes}.

\filbreak

{\samepage
\subsection{Subclasses of Population}

\Defclassv {generational-population} {population}

This class is a subclass of \inxclass{population} which provides explicit support
for the `standard' generational style of GA. The class has no
slots, but methods described elsewhere specialize on this class (see
\inxmethod{regenerate}, page~\pageref{method:regenerate}).
\par}%end samepage
\filbreak

Eventually \geco\ may contain support for other styles of population handling,
possibly including parallel sub-populations, steady-state populations, \etc.

\gap
\filbreak

{\samepage
\bold{Instance Creation and Initialization}

The generic function \inxgeneric{make-population} (see
page~\pageref{method:make-population}) is the \geco\ interface for creation
of instances of \inxclass{population} and its subclasses.
\par}%end samepage

\filbreak

{\samepage
\subsection{Population Mixin Classes}	\label{sec:population-mixin-classes}

\Defclass {maximizing-score-mixin}
\defclass {minimizing-score-mixin}

Neither of these classes has any slots or has special provisions for
instance creation or initialization.
\par}%end samepage

\gap

\filbreak

{\samepage

\bold{Specialized Methods}

Both classes implement methods for the following generic functions:

\Defgeneric maximizing-p {population}
\defmethod maximizing-p {(population \inxclass{maximizing-score-mixin})}
\defmethod maximizing-p {(population \inxclass{minimizing-score-mixin})}
\Defgeneric minimizing-p {population}
\defmethod minimizing-p {(population \inxclass{maximizing-score-mixin})}
\defmethod minimizing-p {(population \inxclass{minimizing-score-mixin})}

These functions permit algorithms to efficiently determine whether the
\arg{population} is minimizing or maximizing.  The \geco-supplied methods
return either \cl{t} or \cl{nil} as appropriate for their class.
\par}%end samepage

%These functions could all be performed by \cl{:allocation :per-class} slots if
%and when they can be implemented portably in Common Lisp.

\filbreak

{\samepage
\Defgeneric convergence-fraction {population}
\defmethod convergence-fraction {(population \inxclass{maximizing-score-mixin})}
\defmethod convergence-fraction {(population \inxclass{minimizing-score-mixin})}

This function returns the convergence-fraction value which should be used for
\arg{population} by the \inxgeneric{converged-p} function. The \geco-supplied
primary methods for both the \inxclass{maximizing-score-mixin} and the
\inxclass{minimizing-score-mixin} classes return $0.95$. These values are not
necessarily the {\em right} numbers in any real sense, but they are probably
reasonable for many applications. Some applications may want to provide different
values, and possibly even adaptive methods for specialized subclasses.
\par}%end samepage

\filbreak

{\samepage
\Defgeneric convergence-threshold-margin {population}
\defmethod convergence-threshold-margin {(population \inxclass{maximizing-score-mixin})}
\defmethod convergence-threshold-margin {(population \inxclass{minimizing-score-mixin})}

This function returns the convergence-threshold-margin value which should be used
for \arg{population} by the \inxgeneric{converged-p} function. The \geco-supplied
primary method provided for the \inxclass{maximizing-score-mixin} class returns
$0.95$, and the method provided for the \inxclass{minimizing-score-mixin} class
returns $0.05$. These values are not necessarily the {\em right} numbers in any real
sense, but they are probably reasonable for many applications. Some applications may
want to provide different values, and possibly even adaptive methods for specialized
subclasses. 
\par}%end samepage

\filbreak

{\samepage
\Defgeneric as-good-as-test {population}
\defmethod as-good-as-test {(population \inxclass{maximizing-score-mixin})}
\defmethod as-good-as-test {(population \inxclass{minimizing-score-mixin})}

This function returns a function of two numeric arguments, which when applied to
\inxslot{score}s from organisms in \arg{population}, indicates whether or not the first
score is as good as the second. The \geco-supplied primary method for the
\inxclass{maximizing-score-mixin} class returns \cl{#'>=}, and the method provided
for the \inxclass{minimizing-score-mixin} class returns \cl{#'<=}.
\par}%end samepage

\filbreak

{\samepage
\Defgeneric better-than-test {population}
\defmethod better-than-test {(population \inxclass{maximizing-score-mixin})}
\defmethod better-than-test {(population \inxclass{minimizing-score-mixin})}

This function returns a function of two numeric arguments, which when applied to
\inxslot{scores} from organisms in \arg{population}, indicates whether or not the first
score is better than the second. The \geco-supplied primary method provided
for the \inxclass{maximizing-score-mixin} class returns \cl{#'>}, and the method
provided for the \inxclass{minimizing-score-mixin} class returns \cl{#'<}.
\par}%end samepage

\filbreak

{\samepage
\Defgeneric best-organism {population}
\defmethod best-organism {(population \inxclass{maximizing-score-mixin})}
\defmethod best-organism {(population \inxclass{minimizing-score-mixin})}

This function returns the best organism in the corresponding population from
population statistics of \arg{population}. The \geco-supplied primary method for the
\inxclass{maximizing-score-mixin} class uses \inxgeneric{max-organism}, and the method
provided for the \inxclass{minimizing-score-mixin} class uses \inxgeneric{min-organism}.
\par}%end samepage

\filbreak

{\samepage
\Defgeneric worst-organism {population}
\defmethod worst-organism {(population \inxclass{maximizing-score-mixin})}
\defmethod worst-organism {(population \inxclass{minimizing-score-mixin})}

This function returns the best organism in the corresponding population from
population statistics of \arg{population}. The \geco-supplied primary method for the
\inxclass{maximizing-score-mixin} class uses \inxgeneric{min-organism}, and the method
provided for the \inxclass{minimizing-score-mixin} class uses \inxgeneric{max-organism}.
\par}%end samepage

\filbreak

{\samepage
\Defgeneric best-organism-accessor {population}
\defmethod best-organism-accessor {(population \inxclass{maximizing-score-mixin})}
\defmethod best-organism-accessor {(population \inxclass{minimizing-score-mixin})}

This function returns a function which can be applied to an instance of the
\inxclass{population-statistics} class of \arg{population} to obtain the best organism in the
corresponding population. The \geco-supplied primary method for the
\inxclass{maximizing-score-mixin} class returns \cl{#'}\inxgeneric{max-organism},
and the method provided for the \inxclass{minimizing-score-mixin} class returns
\cl{#'}\inxgeneric{min-organism}.
\par}%end samepage \filbreak

\filbreak

{\samepage
\Defgeneric worst-organism-accessor {population}
\defmethod worst-organism-accessor {(population \inxclass{maximizing-score-mixin})}
\defmethod worst-organism-accessor {(population \inxclass{minimizing-score-mixin})}

This function returns a function which can be applied to an instance of the
\inxclass{population-statistics} class of \arg{population} to obtain the worst organism in
the corresponding population. The \geco-supplied primary method for the
\inxclass{maximizing-score-mixin} class returns \cl{#'}\inxgeneric{min-organism}, and the
method provided for the \inxclass{minimizing-score-mixin} class returns
\cl{#'}\inxgeneric{max-organism}. \par}%end samepage \filbreak


\section{The Organism Class}

An \concept{organism} is a member of the population which is being evolved by the GA. Typically
an organism represents a single distinct solution to the problem which the GA is set to
solve, although sometimes\footnote{%
%
In some kinds of Learning Classifier Systems \cite{gbml:holland-reitman,gbml:holland-induction},
the so-called `Michigan' approach (for the University of Michigan), each member of a
population represents a rule, and the entire population cooperatively evolves as a ruleset.
By way of contrast, in the `Pitt' approach (for the University of Pittsburg) each member of a
population represents an entire ruleset.
%
} an entire population of organisms cooperate to constitute a solution.
\filbreak

In \geco, an instance of an organism class is a collection of information related to
a population member. This may include an explicit representation of the population
member (the organism's \term{phenotype}), or a coded representation (the
\term{genotype}), or both. An evaluation of the organism (its \term{score}) is also
present, so that the GA can have some way to determine which organisms are better
than others, and to what extent.
\filbreak

Typically, during the operation of the GA, the \term{genetic operator}s
manipulate the organism's genotype, and then that is converted into the
phenotype, which is then evaluated to produce a score. The genotype typically
consists of one or more \term{chromosomes}, which encode the features of the
phenotype. In some GAs the genotype is bypassed, and the \term{genetic operator}s
manipulate the phenotype directly, in which case the genotype is empty. In other
GAs, the organism's score can be determined directly from the genotype, and the
conversion from genotype to phenotype is completely omitted. The phenotype is not
included in the basic \inxclass{organism} class, but as a mixin described later (see
\inxclass{organism-phenotype-mixin}, page~\pageref{class:organism-phenotype-mixin}).

\Defclass {organism}

\filbreak

{\samepage

\gap

\bold{Instance Allocated Slots}

  \Defslotv {population} {nil}
  \definitarg {:population}
  \defaccessor {population}

  Provides a link back to the population to which the organism belongs.
\par}%end samepage

\filbreak
{\samepage
  \Defslotv {genotype} {nil}
  \definitarg {:genotype}
  \defaccessor {genotype}

  A list of zero or more chromosomes, which form an encoded representation
  of the organism. 
\par}%end samepage

\filbreak
{\samepage
  \Defslotv {score} {nil}
  \definitarg {:score}
  \defaccessor {score}

  A (raw) numeric representation of the value of the organism to the GA,
  or (initially) \cl{nil}, indicating that the organism hasn't been evaluated.
\par}%end samepage

\filbreak
{\samepage
  \Defslotv {normalized-score} {nil}
  \definitarg {:normalized-score}
  \defaccessor {normalized-score}

  A normalized version of \inxslot{score}, with respect to the rest of the
population, or \cl{nil}, indicating that the organism either hasn't been
evaluated, or that the scores haven't been normalized.
\par}%end samepage

\gap

\filbreak
{\samepage

\bold{Instance Creation and Initialization}

The generic function \inxgeneric{make-organism} (see page~\pageref{method:make-organism})
is the \geco\ interface for creation of \inxclass{organism} instances.

The initialization for instances of \inxclass{organism} has been extended to
support the following additional initargs:

\Definitarg {:random}

The initialization for organism instances has also been extended to check the
\inxslot{genotype} slot, and if it is null it will create chromosomes for the
organism, using the \inxgeneric{make-chromosomes} function, passing the value of the
\arg{:random} keyword argument. This is intended to support automatic initialization
of the initial population. 
\par}%end samepage

\filbreak
{\samepage
\Definitarg {:no-chromosomes}

When non-\cl{nil}, this initarg suppresses creation of the new
organism's chromosomes.
\par}%end samepage

\gap

\filbreak
{\samepage

\bold{Specialized Methods}

{\samepage
	
% use \Eggeneric so the CLOS generic isn't indexed
\Eggeneric shared-initialize {standard-object slot-names \rest initargs}
\defaftermethod shared-initialize {(organism \inxclass{organism}) slot-names \rest initargs
	\key :random :no-chromosomes}

This method extends the initialization for \inxclass{organism} instances to provide for the 
automatic creation and initialization of the \inxclass{chromosome} instances for the organism.
In the \geco-supplied default method, the \inxgeneric{make-chromosomes} generic function
(see page~\pageref{method:make-chromosomes}) is invoked unless the \inxslot{genotype} slot of 
\arg{organism} has already been initialized, or the \arg{:no-chromosomes} argument is non-\cl{nil}.
If \inxgeneric{make-chromosomes} is called, the \arg{:random} value is passed to it as well.
\par}% end \samepage

% use \Eggeneric so the CLOS generic isn't indexed
\Eggeneric print-object {standard-object stream}
\defmethod print-object {(organism \inxclass{organism}) stream}

This method specializes the standard Common Lisp \inxfun{print-object} function for
organisms. It uses the standard Common Lisp function \inxfun{print-unreadable-object},
includes the type and identity of \arg{organism}, and also causes their
\inxslot{normalized-score} and genotype to be included in the printed
representation.
\par}%end samepage

\filbreak

{\samepage
\Defgeneric copy-organism {organism \key :new-population}	\label{copy-organism:organism}
\defmethod copy-organism {(organism \inxclass{organism})
                          \key (:new-population \cl{(\inxclass{population} \arg{organism})})}

Creates and returns a copy of \arg{organism}, modified to be in the
population specified by the \arg{:new-population} argument. The \term{scores}
(neither \inxslot{score} nor \inxslot{normalized-score}) of \arg{organism} are {\em
not} copied to the new organism (see \inxgeneric{copy-organism-with-score}). The
\geco-supplied primary method will always return an organism of the same class as
\arg{organism}, and uses \inxgeneric{copy-chromosome} to copy each chromosome in the
genotype of \arg{organism} to initialize the genotype of the returned organism.
\par}%end samepage

\filbreak

This function would generally be used to make a copy which will be modified (\eg, by
a \term{genetic operator}), thereby invalidating its score.

%\gap
\filbreak

When using \inxclass{organism-phenotype-mixin}, it is important to be sure that the
\inxslot{phenotype} slot is copied properly when copying an organism. Depending on
the representation of the phenotype, it may or may not be worthwhile to copy it whether
or not it will subsequently be modified by \term{genetic operator}s. In any case, copying
anything more complex than an atom requires consideration of application and
representation specific details.

\filbreak
It may be desirable to define an \cl{:around} method on either
\inxgeneric{copy-organism} or \inxgeneric{copy-organism-with-score} to copy the
phenotype (though it should only be necessary to specialize one of these functions,
not both). Alternatively, a specialized class's primary method (on one of these
functions) could use \cl{call-next-method} to invoke the primary method of class
\inxclass{organism}. If using an \cl{:around} method, don't forget to return the copy.

\filbreak

{\samepage
\Defgeneric copy-organism-with-score {organism \key :new-population}
\defmethod copy-organism-with-score {(organism \inxclass{organism})
                          \key (:new-population \cl{(\inxclass{population} \arg{organism})})}

Creates and returns a copy of the organism in the population specified by the
\arg{:new-population} argument, which defaults to the same population as
\arg{organism}. The \inxslot{score} {\em is} copied to the new organism (see
\inxgeneric{copy-organism}). The \inxslot{normalized-score} is not copied on the
assumption that the new organism will be part of a new population, and therefore the
\inxslot{normalized-score} will need to be recomputed within the context the rest of
the new population. The \geco-supplied primary method uses
\inxgeneric{copy-organism} to create the new organism.
\par}%end samepage

\filbreak
If \arg{organism} is an instance of a class which includes
\inxclass{organism-phenotype-mixin} as one of its superclasses, refer to the
discussion under \inxgeneric{copy-organism}, above, regarding copying the
\inxslot{phenotype} slot.

\filbreak
{\samepage
\Defgeneric make-chromosomes {organism \key :random}    \label{method:make-chromosomes}
\defmethod make-chromosomes {(organism \inxclass{organism}) \key :random}

Creates and returns a complete set of chromosomes for \arg{organism}. If
\arg{:random} is non-\cl{nil}, the chromosomes will have random alleles. The
\geco-supplied primary method makes each chromosome with
\inxgeneric{make-chromosome}, and passes it the \arg{:random} argument. The classes
of the chromosomes are obtained by calling the \inxgeneric{chromosome-classes}
function. The new chromosomes are collected into a list in the same order as the
classes returned from \inxfun{chromosome-classes}, and stored in the \inxslot{genotype}
slot of \arg{organism}, and also returned as the result of the function. This method
makes no attempt to determine the proper size for each chromosome, relying on lower
level methods to determine this (see page \pageref{chromosome:size.initarg}).
\par}%end samepage

\filbreak
{\samepage
\Defgeneric make-chromosome {organism chromosome-class \key :size :random}
\defmethod make-chromosome {(organism \inxclass{organism}) chromosome-class \key :size :random}
	\label{method:make-chromosome}

This function creates and returns an instance of \arg{chromosome-class},
which should be a subclass \inxclass{chromosome} (see Section~\ref{sec:chromosome-class}),
and which will become part of the genotype of \arg{organism}. If
\arg{:random} is non-\cl{nil}, the chromosomes will have random alleles. The
\arg{:size} argument may be used to control the size of the new chromosome, \ie, the
size of its \term{loci vector}. The \arg{organism} argument is present so that the
chromosome can have a back-link to \arg{organism}, and so that subclasses of
\arg{organism} can specialize the chromosome creation process based upon the
organism for which the chromosome is intended. The \geco-supplied primary method
passes the \arg{:size} and \arg{:random} arguments to \inxgeneric{make-instance}.
\par}%end samepage

\filbreak
{\samepage
\Defgeneric chromosome-classes {organism}

This function returns a list of classes to be used to create chromosomes for
instances of the class of of \arg{organism}. The list should contain one class for
each chromosome of \arg{organism}, and the order of the classes will determine the
order of the chromosomes in the organism instances. The GA developer {\em must
implement the primary method} for all subclasses of the class \inxclass{organism}.
\Geco\ does not provide a default primary method specialized on the \inxclass{organism}
class.\footnote{There are comments at the beginning of the {\tt generics.lisp} file
which summarize the functions which should or must be defined to implement a working
GA using \geco.}
\par}%end samepage

\filbreak
{\samepage
\Defgeneric randomize-chromosomes {organism}
\defmethod randomize-chromosomes {(organism \inxclass{organism})}

This function replaces all of the chromosomes belonging to \arg{organism} (if any)
with randomly chosen chromosomes of the appropriate classes for \arg{organism}. The
\geco-supplied primary method determines the appropriate classes for the chromosomes
by calling the function \inxgeneric{chromosome-classes}, and uses the
\inxgeneric{pick-random-alleles} function so that the chosen alleles will be valid
for each locus of each chromosome.
The value returned by this function is not defined.
\par}%end samepage

\filbreak
{\samepage
\Defgeneric genotype-printable-form {organism}
\defmethod genotype-printable-form {(organism \inxclass{organism})}

This function returns a single string which is composed of the printable forms of
each chromosome belonging to \arg{organism}. The \geco-supplied primary method
obtains the printable form of each chromosome using the \cl{~A} format directive
with \cl{format}, and concatenates them, with a space between each chromosome's
string. The chromosomes' strings are in the same order as the chromosomes in the
\inxslot{genotype} slot of \arg{organism}.
\par}%end samepage

\filbreak
{\samepage
\Defgeneric evaluate {thing genetic-plan}
\defaftermethod evaluate {(organism \inxclass{organism}) (plan \inxclass{genetic-plan})}
	\label{evaluate:organism}

This function evaluates \arg{organism} and returns its \term{score}, saving it in
\arg{organism}'s \inxslot{score} slot. Evaluating an organism is generally the most
expensive (computationally) operation a GA performs, therefore saving the
score to prevent future evaluations of the organisms is almost always
worthwhile. For the same reason, it behooves the GA developer to make the evaluation
process as efficient as possible.
\par}%end samepage

\filbreak
The GA developer {\em must implement the primary method} for all subclasses of the
class \inxclass{organism}. \Geco\ does not provide a default primary method specialized on
the \inxclass{organism} class.\footnote{There are comments at the beginning of the {\tt
generics.lisp} file which summarize the functions which should or must be defined to
implement a working GA using \geco.} It is the responsibility of this primary method
to perform the calculation of \arg{organism}'s \term{score}, to store it in
\arg{organism}'s \inxslot{score} slot, and to return it as the result of the
function.

\filbreak
The \geco-supplied \cl{:after} method on class \inxclass{organism} increments the
ecosystem's \inxslot{evaluation-number}.

\filbreak
{\samepage
\defgeneric normalize-score {thing genetic-plan}
	\label{method:normalize-score:organism}
\defmethod normalize-score {(organism \inxclass{organism})
                            \hbox{(genetic-plan \inxclass{genetic-plan})}}

This function computes and returns the normalized value of \arg{organism}'s \inxslot{score},
storing the result in the \inxslot{normalized-score} slot. The invocation of
functions responsible for collection of statistics and
normalization\index{normalization} of scores is handled automatically by
\geco. The \geco-supplied primary method uses values from
\cl{(statistics (population \arg{organism}))} to
calculate the normalized score as follows:

$$\frac{   {\tt score}_{{\sl organism}}
         - {\tt min\ttdash{}score}_{{\sl statistics}}
      }{   {\tt max\ttdash{}score}_{{\sl statistics}}
         - {\tt min\ttdash{}score}_{{\sl statistics}}}$$

}%end samepage

\filbreak

Note that this formula distributes the normalized scores over the interval
[0:1]. This results in normalized scores which are (in general) {\em
not} proportional to fitness, since all organisms with the minimum
fitness will have normalized scores of zero.

\gap

\filbreak
{\samepage
\Defgeneric eidetic {thing-1 thing-2}
\defmethod eidetic {(organism-1 \inxclass{organism}) (organism-2 \inxclass{organism})}

This function is a predicate, returning true if the organism arguments are 
of the same class and have equal chromosomes. The \geco-supplied primary method
determines equality of chromosomes by calling the function \inxgeneric{eidetic} on
each of the chromosomes of the argument organisms.\footnote{I have been questioned
regarding the use of the term \concept{eidetic}, above. From Webster's Third New
International Dictionary of the English Language, Unabridged: ``\ital{eidetic}: of,
relating to, or having the characteristics of eide, essences, forms, or images.
Further: \ital{eide}: plural of eidos, and \ital{eidos}: something that is seen or
intuited: a) in Platonism: idea, b) in Aristotelianism (1): form, essence (2):
species.'' Thus, eidetic can be used to indicate `of the same species,' which is the
essence of my original intent.}
\par}%end samepage

\filbreak
{\samepage
\Defgeneric pick-random-chromosome {organism}
\defmethod pick-random-chromosome {(organism \inxclass{organism})}

This function returns a random chromosome from \arg{organism}. The \geco-supplied
primary method uses \inxgeneric{pick-random-chromosome-index} to pick the chromosome
to return.
\par}%end samepage

\filbreak
{\samepage
\Defgeneric pick-random-chromosome-index {organism}
\defmethod pick-random-chromosome-index {(organism \inxclass{organism})}

This function returns a random index into the list of chromosomes belonging to
\arg{organism}. The \geco-supplied primary method biases the selection by the
relative sizes of each chromosome.
\par}%end samepage

\filbreak
{\samepage
\subsection{Basic Genetic Operators}
	\index{genetic operators!basic organism level}

\Defgeneric mutate-organism {organism \key :chromosome-index :chromosome :locus-index}
\defmethod mutate-organism {(organism \inxclass{organism}) \key
    \hbox{(:chromosome-index \cl{(\inxfun{pick-random-chromosome-index} \arg{organism})})}
    \hbox{(:chromosome \cl{(nth \arg{chromosome-index} (genotype \arg{organism}))})}
    \hbox{(:locus-index \cl{(\inxfun{pick-random-locus-index} \arg{chromosome})})}}

This function \term{mutate}s \arg{organism} randomly. The keyword arguments
can be used to control which particular chromosome to \term{mutate}, and where it should be
\term{mutate}d. The \geco-supplied primary method \term{mutate}s the chromosome of \arg{organism}
indicated by either the \arg{:chromosome-index} argument or the \arg{:chromosome}
argument, picking it randomly otherwise, as shown above. The locus to \term{mutate} is
specified by the \arg{:locus-index} argument, which is otherwise chosen randomly, as
shown above. The actual mutation of the chromosomes is performed by calling the
function \inxgeneric{mutate-chromosome}.
The value returned by this function is not defined.
\par}%end samepage

\filbreak
{\samepage 
\Defgeneric cross-organisms {\hbox{parent-1 parent-2} \hbox{child-1 child-2}
                             \key :chromosome-index :locus-index}
\defmethod cross-organisms {\hbox{(parent-1 \inxclass{organism})} \hbox{(parent-2 \inxclass{organism})}
                            \hbox{(child-1 \inxclass{organism})} \hbox{(child-2 \inxclass{organism})}
    \key
    \hbox{(:chromosome-index \cl{(\inxfun{pick-random-chromosome-index} \arg{parent-1})})}
    \hbox{(:locus-index \cl{(\inxfun{pick-random-locus-index} (nth \arg{chromosome-index}
                                              (\inxfun{genotype} \arg{parent-1})))})}}

This function performs a simple \term{crossover} between the two parent
organisms \arg{parent-1} and \arg{parent-2}, storing the results in the two child
organisms \arg{child-1} and \arg{child-2}. The keyword arguments can be used to
control which particular chromosomes to affect and where. The \geco-supplied primary
method performs the crossover on the chromosome from both parents indicated by
\arg{:chromosome-index} at the locus indicated by \arg{:locus-index}, choosing them
randomly otherwise, as shown above. The actual \term{crossover} of the chromosomes is
performed by calling the function \inxgeneric{cross-chromosomes}.
The value returned by this function is not defined.
\par}%end samepage

\filbreak
{\samepage
\Defgeneric uniform-cross-organisms {\hbox{parent-1 parent-2} \hbox{child-1 child-2}
                             \key :chromosome-index}
\defmethod uniform-cross-organisms {\hbox{(parent-1 \inxclass{organism})} \hbox{(parent-2 \inxclass{organism})}
                                    \hbox{(child-1 \inxclass{organism})} \hbox{(child-2 \inxclass{organism})} \key
    \hbox{(:chromosome-index \cl{(\inxfun{pick-random-chromosome-index} \arg{parent-1})})}
    \hbox{(:bias \cl{0.5})}}

This function performs a uniform \term{crossover}
\cite{ga:uniform-xover,ga:parameterized-uniform-xover,ga:davis-handbook} between the
two parent organisms \arg{parent-1} and \arg{parent-2}, storing the result in the
two child organisms \arg{child-1} and \arg{child-2}. The keyword arguments can be
used to control which particular chromosomes to affect and where. The \geco-supplied
primary method performs the \term{crossover} on the chromosome from both parents indicated
by \arg{:chromosome-index}, choosing it randomly otherwise, as shown above, and
using a bias as indicated \arg{:bias} argument, defaulting as shown above if it is
not specified. The actual \term{crossover} is performed by calling
\inxgeneric{uniform-cross-chromosomes}.
The value returned by this function is not defined.
\par}%end samepage

\filbreak
{\samepage
\Defgeneric 2x-cross-organisms {\hbox{parent-1 parent-2} \hbox{child-1 child-2}
                             \key :chromosome-index :locus-index1 :locus-index2}
\defmethod 2x-cross-organisms {\hbox{(parent-1 \inxclass{organism})} \hbox{(parent-2 \inxclass{organism})}
                                    \hbox{(child-1 \inxclass{organism})} \hbox{(child-2 \inxclass{organism})}
 \key
    \hbox{(:chromosome-index \cl{(\inxfun{pick-random-chromosome-index} \arg{parent-1})})}
    \hbox{(:locus-index1 \cl{(\inxfun{pick-random-locus-index}
                              (nth \arg{chromosome-index} (\inxfun{genotype} \arg{parent-1})))})}
    \hbox{(:locus-index2 \cl{(\inxfun{pick-random-locus-index}
                              (nth \arg{chromosome-index} (\inxfun{genotype} \arg{parent-1})))})}}
 
This function performs a two-point \term{crossover} between the two
parent organisms \arg{parent-1} and \arg{parent-2}, storing the result in the two
child organisms \arg{child-1} and \arg{child-2}. The keyword arguments can be used
to control which particular chromosomes to affect and where. The \geco-supplied
primary method performs the \term{crossover} on the chromosome from both parents indicated
by \arg{:chromosome-index}, choosing it randomly otherwise, as shown above. The
actual \term{crossover} of chromosomes is performed between the two sites specified by
\arg{:locus-index1} and \arg{:locus-index2} (which default as shown above, to
randomly chosen sites) by calling the function \inxgeneric{2x-cross-chromosomes}.
The value returned by this function is not defined.
\par}%end samepage

\filbreak
{\samepage
\Defgeneric r3-cross-organisms {\hbox{parent-1 parent-2} \hbox{child-1 child-2}
                             \key :chromosome-index :allele-test}
\label{method:r3-cross-organisms}
\defmethod r3-cross-organisms {\hbox{(parent-1 \inxclass{organism})} \hbox{(parent-2 \inxclass{organism})}
                               \hbox{(child-1 \inxclass{organism})} \hbox{(child-2 \inxclass{organism})}
    \key \hbox{(:chromosome-index \cl{(\inxfun{pick-random-chromosome-index} \arg{parent-1})})}
         \hbox{(:allele-test \cl{\#'eql})}}

This function performs a random respectful recombination \term{crossover}
\cite{ga:radcliffe92-11,ga:radcliffe92-04} between the two parent organisms
\arg{parent-1} and \arg{parent-2}, storing the result in the two child organisms
\arg{child-1} and \arg{child-2}. The keyword arguments can be used to control which
particular chromosomes to affect and where. The \geco-supplied primary method
performs the \term{crossover} on the chromosome from both parents indicated by
\arg{:chromosome-index}, choosing it randomly otherwise, as shown above, and using
the \arg{:allele-test} argument to specify a function to tell when two alleles are
the same, defaulting as shown above if unspecified. The actual \term{crossover} of
chromosomes is performed by calling the function \inxgeneric{r3-cross-chromosomes}.
The value returned by this function is not defined.
\par}%end samepage

\filbreak
{\samepage
\Defgeneric pmx-cross-organisms {\hbox{parent-1 parent-2} \hbox{child-1 child-2}
                         \key :allele-test :chromosome-index :locus-index1 :locus-index2}
\defmethod pmx-cross-organisms {\hbox{(parent-1 \inxclass{organism})}
                                \hbox{(parent-2 \inxclass{organism})}
                                \hbox{(child-1 \inxclass{organism})}
                                \hbox{(child-2 \inxclass{organism})}
    \key \hbox{(:allele-test \cl{\#'eql})}
         \hbox{(:chromosome-index \cl{(\inxfun{pick-random-chromosome-index} \arg{parent-1})})}
         \hbox{(:locus-index1 \cl{(\inxfun{pick-random-locus-index}
                                   (nth \arg{chromosome-index} (\inxfun{genotype} \arg{parent-1})))})}
         \hbox{(:locus-index2 \cl{(pick-random-locus-index
                                   (nth \arg{chromosome-index} (\inxfun{genotype} \arg{parent-1})))})}}

This function performs a partially mapped \term{crossover} (PMX)
\cite{ga:goldberg} between the two parent organisms \arg{parent-1} and
\arg{parent-2}, storing the result in the two child organisms \arg{child-1} and
\arg{child-2}. The keyword arguments can be used to control which particular
chromosomes to affect and where. The \geco-supplied primary method performs the
\term{crossover} on the chromosome from both parents indicated by \arg{:chromosome-index},
which should indicate a \inxclass{sequence-chromosome}, choosing it randomly
otherwise as shown above, and using the \arg{:allele-test} argument to specify a
function to tell when two alleles are the same, defaulting as shown above if
unspecified. Note that if \arg{:chromosome-index} is not specified, all the
chromosomes should be sequence chromosomes, since PMX is only defined for sequence
chromosomes, and the chromosome will be chosen randomly. The actual \term{crossover} of
chromosomes is performed between the two sites specified by \arg{:locus-index1} and
\arg{:locus-index2} (which default as shown above, to randomly chosen sites) by
calling the function \inxgeneric{pmx-cross-chromosomes}.
The value returned by this function is not defined.
\par}%end samepage

\subsection{Organism Mixin Classes}	\label{sec:organism-mixin-classes}

Presently, there is only one mixin class intended to be used with organism classes.

\filbreak
{\samepage
\Defclass {organism-phenotype-mixin} \label{class:organism-phenotype-mixin}

This class is intended to be mixed with organism classes which need to have a
\term{phenotype}\index{phenotype} represented for each \term{organism}. It is an
abstract (non-instantiable) class.
\par}%end samepage

\filbreak
Often it is necessary to decode the \term{genotype} into a phenotype before the
organism can be evaluated and assigned a score. Also, some GAs bypass the encoded
genotype and use only the phenotype, requiring specially crafted \term{genetic operator}s which
manipulate the phenotype directly.

\filbreak
Note that users of this class should review the discussion regarding copying the
\inxslot{phenotype} slot included in the description of
\inxgeneric{copy-organism} (page \pageref{copy-organism:organism}).

\gap
  
\filbreak
{\samepage

\bold{Instance Allocated Slots}

  \Defslot {phenotype}
  \definitarg {:phenotype}
  \defaccessor {phenotype}

  An explicit representation of the organism, \ie, its realization.
\par}%end samepage

\gap

\filbreak
{\samepage

\bold{Specialized Methods}

\Defgeneric decode {organism}

This function converts \arg{organism}'s \inxslot{genotype} to its \term{phenotype},
and stores it in the \inxslot{phenotype} slot. \Geco{} automatically invokes
\inxgeneric{decode}, when appropriate, for instances of
\inxclass{organism-phenotype-mixin} subclasses (see \inxgeneric{evaluate}, above).
The value returned by this function is not defined.
\par}%end samepage

\filbreak
The GA developer {\em must implement the primary method} for all subclasses of the
class \inxclass{organism-phenotype-mixin} which performs the decoding operation
required by the GA application. \Geco\ does not provide a default primary method
specialized on the \inxclass{organism-phenotype-mixin} class.\footnote{There are comments
at the beginning of the {\tt generics.lisp} file which summarize the functions which
should or must be defined to implement a working GA using \geco.}

\filbreak

{\samepage
\Defgeneric evaluate {thing genetic-plan}
\defbeforemethod evaluate {(organism \inxclass{organism-phenotype-mixin}) (plan \inxclass{genetic-plan})}

This function evaluates \arg{organism} and returns its \inxslot{score}. \Geco\
provides a \cl{:before} method on class \inxclass{organism-phenotype-mixin}, which
invokes the generic function \inxgeneric{decode} on \arg{organism}, so that the
genotype will be decoded into a phenotype which can be used by the primary method of
\inxgeneric{evaluate} (see page \pageref{evaluate:organism}).
\par}%end samepage

\filbreak


\section{The Chromosome Class} \label{sec:chromosome-class}

An organism's \term{genotype}\index{genotype} is made up of one or more
\term{chromosomes}, which contain the encoded genetic representation of what makes the
organism different from other organisms. The actual encoding scheme used may vary between
different types of organisms, and even between chromosomes of a single type of organism.
\Geco{} implements much of the functionality of chromosomes independently of the type of
encoding used by the chromosome, but also provides some explicit support for some
of the most common kinds of chromosomes via subclasses of the class \inxclass{chromosome} (see
Section~\ref{sec:subclasses-of-chromosome}).

\filbreak
{\samepage
\Defclass {chromosome}

This class is the basic class upon which all chromosome classes are
based.  It is an abstract (non-instantiable) class.
\par}%end samepage

\gap

\filbreak
{\samepage

\bold{Instance Allocated Slots}

\Defslotv {organism} {nil}
\definitarg {:organism}
\defaccessor {organism}

This slot points back to the organism to which the chromosome belongs.
\par}%end samepage

\filbreak
{\samepage
\Defslot {loci}
\definitarg {:loci}
\defaccessor {loci}

This slot contains the \concept{loci-vector}, which is generally a simple,
one-dimensional array, whose elements jointly encode the genetic information of the
chromosome. Note that the individual loci need not all be of the same type, though
they usually are.
\par}%end samepage

\gap

\filbreak
{\samepage

\bold{Instance Creation and Initialization}

The generic function \inxgeneric{make-chromosome} (see
page~\pageref{method:make-chromosome}) is the \geco\ interface for creation
of \inxclass{chromosome} instances.

The initialization for chromosome instances has been extended to support the following
additional initargs:

\Definitarg {:random}

A non-\cl{nil} value for this initarg indicates that each locus should be
initialized to a random allele. The value of this keyword is passed to the
\inxgeneric{make-loci-vector} function, and is intended to support automatic
generation of the initial population\index{population!initialization}, and/or
creation of random organisms which could be added to a population to increase or
restore its diversity.
\par}%end samepage

\filbreak
{\samepage
\Definitarg {:size}	\label{chromosome:size.initarg}

The value of this keyword determines the size of the chromosome, \ie, the size of
the \term{loci vector} of the chromosome. If its value is \cl{nil}, or it is
unspecified, the function \inxgeneric{size} is invoked on the new instance.
Specialization of the \inxfun{size} function for the instantiable chromosome class is
the normal way to control the size of chromosome instances.
\par}%end samepage

\gap
  
\filbreak
{\samepage
	

\bold{Specialized Methods}

% use \Eggeneric so the CLOS generic isn't indexed
\Eggeneric shared-initialize {standard-object slot-names \rest initargs}
\defaftermethod shared-initialize {(self \inxclass{chromosome}) slot-names \rest initargs
	\key :size :organism :random}

This method extends the initialization for \inxclass{chromosome} instances to provide for the 
automatic linking of the \inxclass{chromosome} \arg{self} back to it's \inxclass{organism} instance,
and, if the \inxslot{loci} is unbound, creation and initialization of the loci vector via
\inxgeneric{make-loci-vector} (see below).
If it is necessary to invoke \inxgeneric{make-loci-vector}, and the \arg{:size} argument is
specified, then that is the size used, otherwise, the \inxslot{size} slot of the \inxclass{organism}
instance is used; also, the value of the \arg{:random} keyword is passed (which defaults to \cl{nil} if
it is not specified).\
\par}% end \samepage

\filbreak

{\samepage

\Defgeneric make-loci-vector {chromosome size \key :random}
\defmethod make-loci-vector {(chromosome \inxclass{chromosome}) size \key \allow}
\defaroundmethod make-loci-vector {(chromosome \inxclass{chromosome}) size
                                   \key :random}
	\label{chromosome:make-loci-vector}
This function creates and returns a \term{loci-vector} for \arg{chromosome} of size \arg{size}
and puts it into the \inxslot{loci} slot of \arg{chromosome}. The \geco-supplied
primary method creates an array whose element-type is \cl{fixnum}, with all the
elements initialized to zero (0). The \geco-supplied \cl{:around} method examines
the value of the \arg{:random} argument, and if it is non-\cl{nil} passes
\arg{chromosome} to \inxgeneric{pick-random-alleles}. Since the \cl{:around} method
processes the \arg{:random} argument, the primary method uses the \key \allow
sequence to avoid processing it.
\par}%end samepage

\filbreak
{\samepage
\Defgeneric locus-arity {chromosome locus-index}

This function returns the number of allele values which are allowed at the locus
indicated by \arg{locus-index} in \arg{chromosome}. No primary method is predefined
for the general class \arg{chromosome}, but one {\em must be implemented} for any
instantiable chromosome class.\footnote{There are comments at the beginning of the
{\tt generics.lisp} file which summarize the functions which should or must be
defined to implement a working GA using \geco.} Note that locus arity may be a
function of \arg{locus-index}, though this is relatively uncommon.
\par}%end samepage

\filbreak
{\samepage
\Defgeneric copy-chromosome {chromosome owner-organism}
\defmethod copy-chromosome {(chromosome \inxclass{chromosome}) owner-organism}

This function returns a copy of \arg{chromosome}, setting the \inxslot{organism}
slot of the new chromosome to \arg{owner-organism}. The \geco-supplied primary
method makes the copy using \inxgeneric{make-chromosome}, passing it the class of
\arg{chromosome}, and initializes the \term{loci vector} by assigning each locus the
same value as the corresponding locus of \arg{chromosome}. Note that this method of
copying the alleles may not be appropriate for some chromosome classes, \eg, ones
whose loci vectors are not atomic, and which may be manipulated (changed) in ways
which might affect more than one organism.
\par}%end samepage

\filbreak
{\samepage
% use \Eggeneric so the CLOS generic isn't indexed
\Eggeneric print-object {standard-object stream}
\defmethod print-object {(chromosome \inxclass{chromosome}) stream}

This method specializes the standard Common Lisp \inxfun{print-object} function for
chromosomes. It uses the standard Common Lisp function \inxfun{print-unreadable-object},
includes the type and identity of \arg{chromosome}, and also uses
\inxgeneric{loci-printable-form} to include a representation of the alleles of
\arg{chromosome}.
\par}%end samepage

\filbreak
{\samepage
\Defgeneric eidetic {thing-1 thing-2}
\defmethod eidetic {(chromosome-1 \inxclass{chromosome}) (chromosome-2 \inxclass{chromosome})}

This function is a predicate, returning true (non-\cl{nil}) if the arguments are the
same. In the case of instances of chromosome classes, being the same means that they
are of the same class, have the same size, and the same alleles at corresponding
loci in their \term{loci vectors}. The \geco-supplied primary method compares the
alleles (which are expected to be \term{allele codes}, see 
Section~\ref{allele-coding}, page~\pageref{allele-coding}) using \cl{#'=}.
\par}%end samepage

\filbreak
{\samepage
\Defgeneric size {thing}
\defmethod size {(chromosome \inxclass{chromosome})}

This function returns the size of its argument in whatever units are appropriate. The
\geco-supplied primary method for \inxclass{chromosome} returns the size of the
\term{loci vector} belonging to \arg{chromosome}.
\par}%end samepage

\filbreak
{\samepage
\Defgeneric pick-random-locus-index {chromosome}
\defmethod pick-random-locus-index {(chromosome \inxclass{chromosome})}

This function returns a random index into the \term{loci vector} of
\arg{chromosome}. The \geco-supplied primary method calls
\inxfun{geco-random-integer} with the size of \arg{chromosome}.
\par}%end samepage

\filbreak
{\samepage
\Defgeneric hamming-distance {chromosome-1 chromosome-2}
\defmethod hamming-distance {(chromosome-1 \inxclass{chromosome}) (chromosome-2 \inxclass{chromosome})}

This function returns the count of the number of loci in the two arguments
which have different alleles at corresponding loci. The \geco-supplied
primary method compares the number of loci which are in \arg{chromosome-1},
and uses \cl{#'=} to compare the \term{allele codes} (see
Section~\ref{allele-coding}, below). It is an error
if the entire part of \arg{chromosome-2} designated is not within its
\term{loci vector}, \ie, if an invalid locus index is implied by the
arguments.
\par}%end samepage


\subsection{Allele Coding: Codes vs.\ Values}	\label{allele-coding}
	\index{allele values}\index{allele codes}

Fixnums are chosen as the default type for \term{loci-vector} elements because they
can frequently be stored more efficiently than general lisp values, particularly
when there are only a small number of alleles per locus. To make this choice more
generally useful, \geco\ interprets the values stored in loci-vector elements as
\concept{allele codes}, as opposed to \concept{allele values}. This allows a
straightforward conversion between these fixnums and the actual alleles via simple
table lookups, \ie, the table contains the allele values and is indexed by the
allele code. \Geco\ supports this translation directly via the generic functions
\inxgeneric{allele-values} and \inxgeneric{allele-code-to-value}. \Geco\ also
supports conversion of the allele codes to printable form via the generic functions
\inxgeneric{printable-allele-values}, \inxgeneric{loci-printable-form} and
\inxgeneric{locus-printable-form}. These functions are describe below. Note that the
descriptions of some functions may gloss over the distinction between allele codes
and allele values, referring to either of them simply as alleles, but it should be
clear from context which is being manipulated.

\filbreak
{\samepage
\Defgeneric pick-random-alleles {chromosome}
\defmethod pick-random-alleles {(chromosome \inxclass{chromosome})}

This function initializes the loci of \arg{chromosome}'s \inxslot{loci-vector} to
random alleles. The \geco-supplied primary method calls \inxfun{pick-random-allele} for
each locus to obtain its new allele.
\par}%end samepage

\filbreak
{\samepage
\Defgeneric pick-random-allele {chromosome locus-index}
\defmethod pick-random-allele {(chromosome \inxclass{chromosome}) locus-index}

This function returns a random \term{allele code} for the indicated locus of
\arg{chromosome}. The \geco-supplied primary method selects a random number in the
proper range by calling \inxfun{geco-random-integer} with the value returned by the
\inxgeneric{locus-arity} function for the indicated chromosome and locus-index.
\par}%end samepage

\filbreak
{\samepage
\Defgeneric allele-code-to-value {chromosome locus-index allele-code}
\defmethod allele-code-to-value {(chromosome \inxclass{chromosome}) locus-index allele-code}

This function converts \arg{allele-code} to an \term{allele value} (see
page~\pageref{allele-coding}), returning the \term{allele value}.
The \arg{chromosome} and \arg{locus-index} arguments
permit different loci of different chromosomes to have different mappings (codings)
between allele codes and allele values. In particular, this permits different
chromosomes/loci to have different arity. The \geco-supplied primary method uses
\cl{aref} to index into the array returned by \inxgeneric{allele-values}.
\par}%end samepage

\filbreak
{\samepage
\Defgeneric allele-values {chromosome locus-index}

This function returns a vector of \term{allele values}, which may be used to convert the
\term{allele codes} used in \term{loci vectors}. \Geco\ {\em does not} implement a
primary method for this function for the \inxclass{chromosome} class. Instantiable chromosome
classes should implement this method based on the genetic representation they
use.\footnote{There are comments at the beginning of the {\tt generics.lisp} file which
summarize the functions which should or must be defined to implement a working GA using
\geco.}
\par}%end samepage

\filbreak
Note that it is generally preferable to use the function
\inxgeneric{allele-code-to-value}, rather than indexing into the vector returned by
this function, since the implementation may permit a more efficient implementation
than is supported by this general mechanism (\eg, for subclasses of
\inxclass{binary-chromosome}).

\filbreak
{\samepage
\Defgeneric printable-allele-values {chromosome locus-index}

This function returns a vector of characters indexed by \term{allele code} to
generate a printable representation for a chromosome. \Geco\ \ital{does not}
implement a primary method for this function for the \inxclass{chromosome} class.
Instantiable chromosome classes should implement this method based on the genetic
representation they use.\footnote{There are comments at the beginning of the {\tt
generics.lisp} file which summarize the functions which should or must be defined to
implement a working GA using \geco.}
\par}%end samepage

\filbreak
Note that it is generally preferable to call one of the functions
\inxgeneric{loci-printable-form} or \inxgeneric{locus-printable-form}, rather than
indexing into the vector returned by this function, since their implementation may
permit a more efficient implementation than is supported by this general mechanism
(\eg, for subclasses of \inxclass{binary-chromosome}).

\filbreak
{\samepage
\Defgeneric loci-printable-form {chromosome}
\defmethod loci-printable-form {(chromosome \inxclass{chromosome})}

This function returns a string which is a printable representation of the
\inxslot{loci-vector} of \arg{chromosome}. The \geco-supplied primary method
constructs a string whose length is the size of \arg{chromosome}, and whose
characters represent the alleles of the chromosome on a one-for-one basis, with the
first character corresponding to the first locus' allele, \etc. The characters
representing each locus' allele are determined by calling
\inxgeneric{locus-printable-form}.
\par}%end samepage

\filbreak
{\samepage
\Defgeneric locus-printable-form {chromosome locus-index}
\defmethod locus-printable-form {(chromosome \inxclass{chromosome}) locus-index}

This function returns the character which represents the allele at \arg{locus-index}
in \arg{chromosome}. The \geco-supplied primary method uses \cl{aref} to index into
the vector returned by \inxgeneric{printable-allele-values} with the \term{allele
code} found at \arg{locus-index} in \arg{chromosome}. If the allele code is not a
valid index for \inxfun{printable-allele-values}, return \cl{#\\?}.
\par}%end samepage

\filbreak
{\samepage
\Defgeneric locus {chromosome locus-index}
\defmethod locus {(chromosome \inxclass{chromosome}) locus-index}

This function returns the \term{allele code} at \arg{locus-index} in the
\inxslot{loci} of \arg{chromosome}.
\par}%end samepage

\filbreak
{\samepage
\Defgeneric {(setf locus)} {allele-code chromosome locus-index}
\defmethod {(setf locus)} {allele-code (chromosome \inxclass{chromosome}) locus-index}

This function stores the \arg{allele-code} into the locus indicated by \arg{locus-index} in
the \term{loci vector} of \arg{chromosome}.
\par}%end samepage

\filbreak
Note that Common Lisp has rather non-intuitive ordering for the
arguments for \cl{setf} functions and methods.  An example of proper
invocation is: \begin{clcode}(setf (locus chromosome locus#) allele-code)\end{clcode}

\filbreak
{\samepage
\Defgeneric count-allele-codes {chromosome from-index loci-to-count allele-code}
\defmethod count-allele-codes {(chromosome \inxclass{chromosome}) from-index loci-to-count allele-code}

This function returns the count of the number of loci in part of the
\arg{chromosome} which have \arg{allele-code} in them. The part of \arg{chromosome}
in which the count is conducted are specified as starting at the locus whose index
is \arg{from-index} and which is \arg{loci-to-count} long. The \geco-supplied
primary method compares the alleles to \arg{allele-code} using \cl{#'=}. It is an
error if the entire part of \arg{chromosome} designated is not within the \term{loci
vector}, \ie, if an invalid locus index is implied by the arguments \arg{from-index}
and \arg{loci-to-count}.
\par}%end samepage

\filbreak
{\samepage
\subsection{Basic Chromosomal Genetic Operators}\index{genetic operators!basic chromosomes}

\Defgeneric mutate-chromosome {chromosome locus-index}
\defmethod mutate-chromosome {(chromosome \inxclass{chromosome}) locus-index}

This function \term{mutate}s \arg{chromosome} at the locus \arg{locus-index}.
The \geco-supplied primary method uses \inxgeneric{pick-random-allele} to choose
the new \term{allele code} for the locus.

Note that for instances of subclasses of \inxclass{binary-chromosome},
this implementation will produce on average one mutation for every two
invocations of this function, since half the time the randomly chosen
allele will be the same as the current allele at the indicated locus.
The value returned by this function is not defined.
\par}%end samepage

\filbreak
{\samepage
\Defgeneric cross-chromosomes {\hbox{parent-1 parent-2}
                               \hbox{child-1 child-2}
                               locus-index}
\defmethod cross-chromosomes {\hbox{(parent-1 \inxclass{chromosome})}
                              \hbox{(parent-2 \inxclass{chromosome})}
                              \hbox{(child-1 \inxclass{chromosome})}
                              \hbox{(child-2 \inxclass{chromosome})}
                              locus-index}

This function performs a simple \term{crossover} operation between the two
parent chromosomes, storing the results in the two child chromosomes, using
\arg{locus-index} as a control parameter for the \term{crossover}. The \geco-supplied
primary method performs a conventional one-point \term{crossover}, assumes all the
chromosomes are the same size and of compatible classes, the \arg{child-1} receives
\arg{locus-index} alleles from \arg{parent-1}, and the remaining alleles from
\arg{parent-2}; \arg{child-2} gets its alleles in an analogous manner.
The value returned by this function is not defined.
\par}%end samepage

\filbreak
{\samepage
\Defgeneric uniform-cross-chromosomes {\hbox{parent-1 parent-2}
                               \hbox{child-1 child-2}
                               \key :bias}
\defmethod uniform-cross-chromosomes {\hbox{(parent-1 \inxclass{chromosome})}
                              \hbox{(parent-2 \inxclass{chromosome})}
                              \hbox{(child-1 \inxclass{chromosome})}
                              \hbox{(child-2 \inxclass{chromosome})}
                              \key (:bias \cl{0.5})}

This function performs a uniform \term{crossover}
\cite{ga:uniform-xover,ga:parameterized-uniform-xover,ga:davis-handbook} operation
between the two parent chromosomes, storing the results in the two child
chromosomes, using the \arg{:bias} argument as a control parameter for the
\term{crossover}. The \geco-supplied primary method performs a conventional uniform
\term{crossover}, assumes all the chromosomes are the same size and of compatible classes,
\arg{child-1} statistically receives a fraction of the alleles specified by
\arg{:bias} from \arg{parent-1}, and the remaining alleles from \arg{parent-2};
\arg{child-2} gets its alleles in an analogous manner.
The value returned by this function is not defined.
\par}%end samepage

\filbreak
{\samepage
\Defgeneric 2x-cross-chromosomes {\hbox{parent-1 parent-2}
                               \hbox{child-1 child-2}
                               \hbox{locus-index1 locus-index2}}
\defmethod 2x-cross-chromosomes {\hbox{(parent-1 \inxclass{chromosome})}
                              \hbox{(parent-2 \inxclass{chromosome})}
                              \hbox{(child-1 \inxclass{chromosome})}
                              \hbox{(child-2 \inxclass{chromosome})}
                              \hbox{locus-index1 locus-index2}}

This function performs a two-point \term{crossover} operation between the
two parent chromosomes, storing the results in the two child chromosomes. The
\geco-supplied primary method performs a conventional two-point \term{crossover}, assumes
all the chromosomes are the same size and of compatible classes. Alleles between
\arg{locus-1} and \arg{locus-2} are copied from from \arg{parent-1} to
\arg{child-2}, and the remaining alleles from \arg{parent-2}; \arg{child-1} gets
its alleles in an analogous manner. If \arg{locus-1} is greater than \arg{locus-2},
the copy operation wraps around from the end of the chromosome back to its
beginning, then copies from the beginning to \arg{locus-2}.
The value returned by this function is not defined.
\par}%end samepage

\filbreak
{\samepage
\Defgeneric swap-alleles {chromosome \key :locus-index :locus-index2}
\label{method:swap-alleles}
\defmethod swap-alleles {(chromosome \inxclass{chromosome}) \key
                         \hbox{(:locus-index \cl{(\inxfun{pick-random-locus-index} \arg{chromosome})})}
                         \hbox{(:locus-index2 \cl{(mod (1+ \arg{locus-index})
                                                 (size \arg{chromosome}))})}}

This function swaps alleles between two loci of \arg{chromosome}. The two loci to
swap are indicated by the arguments \arg{:locus-index} and \arg{:locus-index2}. The
\geco-supplied primary method allows the keyword arguments to default as shown
above.
The value returned by this function is not defined.
\par}%end samepage

\filbreak
{\samepage
\Defgeneric scramble-alleles {chromosome}
\defmethod scramble-alleles {(chromosome \inxclass{chromosome})}

This function randomly rearranges the alleles of \arg{chromosome}. The
\arg{chromosome} will have the same set of \term{allele codes} both before and after
the operation, but they will appear in a different permutation on the loci. Note
that this operator should not be applied to chromosomes for which the arity of all
loci is not the same.
The value returned by this function is not defined.
\par}%end samepage
\filbreak


\subsection{Subclasses of Chromosome}
	\label{sec:subclasses-of-chromosome}

\Geco\ provides some support for some of the more common kinds of
chromosomes.  Presently, this includes:
\begin{itemize}
  \item Binary chromosomes
  \item Sequence chromosomes
\end{itemize}

This section also describes some support provided for decoding binary
coded chromosomes.

\filbreak
{\samepage
\subsubsection{Binary Chromosomes}

\Defclassv {binary-chromosome} {chromosome}

Binary chromosomes are a subclass of \inxclass{chromosome} whose alleles are always
chosen from the set \{0, 1\}. This restriction allows them to be represented
more efficiently, and specialized methods can be provided which process them
somewhat more efficiently than the more general case.
\par}%end samepage

\filbreak
Note that \inxclass{binary-chromosome} is still an abstract (non-instantiable)
class, since the size of the chromosome is left unspecified.

\gap\filbreak

This class has no additional slots beyond those defined for the
\inxclass{chromosome} class.

\gap\filbreak
{\samepage

\bold{Instance Creation and Initialization}

The generic function \inxgeneric{make-chromosome} (see
page~\pageref{method:make-chromosome}) is the \geco\ interface for creation
of \inxclass{chromosome} instances.
\par}%end samepage

\gap\filbreak
{\samepage

\bold{Specialized Methods}

\Defgeneric locus-arity {chromosome locus-index}
\defmethod locus-arity {(chromosome \inxclass{binary-chromosome}) locus-index}

This function returns the number of \term{allele values} which are allowed at
the locus indicated by \arg{locus-index} in \arg{chromosome}.  The
\geco-supplied primary method always returns $2$, regardless of the
value of \arg{locus-index}.
\par}%end samepage

\filbreak
{\samepage
\Defgeneric allele-code-to-value {chromosome locus-index allele-index}
\defmethod allele-code-to-value {(chromosome \inxclass{binary-chromosome}) locus-index allele-index}

This function converts \arg{allele-code} to an \term{allele value} (see the
discussion on allele coding in Section~\ref{allele-coding}). The \geco-supplied
primary method simply returns the \term{allele code}, since the value and the code
are the same.
\par}%end samepage

\filbreak
{\samepage
\Defgeneric allele-values {chromosome locus-index}
\defmethod allele-values {(chromosome \inxclass{binary-chromosome}) locus-index}

This function returns a vector of \term{allele values}, which may be used to convert the
\term{allele codes} used in \term{loci-vector}s. The \geco-supplied primary method for
\inxclass{binary-chromosome} always returns the vector \cl{#(0 1)}, regardless of the
value of \arg{locus-index}.
\par}%end samepage

\filbreak
{\samepage
\Defgeneric printable-allele-values {chromosome locus-index}
\defmethod printable-allele-values {(chromosome \inxclass{binary-chromosome}) locus-index}

This function returns a vector of characters which may be indexed by \term{allele code}
to generate a printable representation of \arg{chromosome}. The \geco-supplied primary
method for \inxclass{binary-chromosome} always returns the vector \cl{#(#\\0 #\\1)},
regardless of the value of \arg{locus-index}.
\par}%end samepage

\filbreak
{\samepage
\Defgeneric make-loci-vector {chromosome size \key :random}
\defmethod make-loci-vector {(chromosome \inxclass{binary-chromosome}) size \key \allow}

This function creates and returns a \term{loci vector} for \arg{chromosome} of size \arg{size}
and puts it into the \inxslot{loci} slot of \arg{chromosome}. The \geco-supplied
primary method creates an array whose element-type is \cl{bit}, and with all the
elements initialized to zero (0). Since the inherited \cl{:around} method
(page~\pageref{chromosome:make-loci-vector}) processes the \arg{:random} argument,
the primary method uses the \key \allow sequence to avoid processing it.
\par}%end samepage

\filbreak
{\samepage
\subsubsection{Binary Chromosome Decoding}

\Defgeneric decode-binary-loci-value {chromosome from-index loci-to-decode}
\defmethod decode-binary-loci-value {(chromosome \inxclass{binary-chromosome})
                                     from-index loci-to-decode}

This function returns the numeric value encoded by the loci of \arg{chromosome}
which start at the locus indexed by \arg{from-index} and are \arg{loci-to-decode} in
length. The \geco-supplied primary method treats the loci as an unsigned binary
coded bit string, with the most significant bits having the lower indices in the
\term{loci vector}.
\par}%end samepage
\filbreak


\subsubsection{Gray Code Translation}\index{gray code translation}
	\label{section:gray-code-translation}

Sometimes it is advantageous to treat a binary coded value as if it were encoded
using a gray code scheme\cite{ga:gray-code-bias}. \Geco\ provides a special class
whose instances can be used for quickly decoding (or encoding) gray coded binary
values.

The conversion scheme implemented by \geco\ is based on an implementation in \cl{C} by
Larry Yaeger \verb|<larryy@apple.com>|, which was published in the GA-List Digest v6n5
\hbox{(\verb|GA-List@AIC.NRL.Navy.Mil|).}

\filbreak
{\samepage
\Defclass {gray-code-translation}

A class whose instances support translation between standard binary and gray
coded integer values for a specified number of bits.
\par}%end samepage

\gap

\filbreak
{\samepage

\bold{Instance Allocated Slots}

\Defslot {number-of-bits}
\definitarg {:number-of-bits}
\defaccessor {number-of-bits}

This slot specifies the number of bits in the bit string which will be encoded or
decoded. This initarg should be specified when an instance of
\inxclass{gray-code-translation} is created for proper initialization of the
instance.
\par}%end samepage

\filbreak
{\samepage
\Defslot {b2g-map}
\defaccessor {b2g-map}

\Defslot {g2b-map}
\defaccessor {g2b-map}

When the \inxinitarg{:number-of-bits} initarg is specified at instance creation time,
these two slots will be initialized to bit maps which are used by the conversion methods
described below.
\par}%end samepage

\gap\filbreak

{\samepage
\bold{Instance Creation and Initialization}

No special functions for the creation of \inxclass{gray-code-translation} instances have been defined,
since \inxgeneric{make-instance} and the standard \term{CLOS} protocol it follows provide all
the necessary functionality.

Note that the \inxinitarg{:number-of-bits} initarg should be specified when an instance of
\inxclass{gray-code-translation} is created for proper initialization of the instance.
\par}%end samepage

\gap

\filbreak
{\samepage


\bold{Specialized Methods}

% use \Eggeneric so the CLOS generic isn't indexed
\Eggeneric shared-initialize {standard-object slot-names \rest initargs}
\defaftermethod shared-initialize {(self \inxclass{gray-code-translation}) slot-names \rest initargs
	\key :number-of-bits :number-of-bits-supplied-p}

This method extends the initialization for \inxclass{gray-code-translation} instances to provide for
proper initialization of the \inxclass{gray-code-translation} class-specific slots.
If the \arg{:number-of-bits} initarg is specified, it is a number, and either \inxslot{b2g-map} or
\inxslot{g2b-map} is unbound, then the instance \arg{self} will be initialized; otherwise, the map slots
will be marked as unbound.
\par}% end \samepage

\filbreak

{\samepage

\Defgeneric gray2bin {translation-instance gray-coded-value}
\defmethod gray2bin {(translation-instance \inxclass{gray-code-translation}) value}

This function uses \arg{translation-instance} to convert the gray coded
\arg{value} to its binary coded equivalent, which it returns.
\par}%end samepage

\filbreak
{\samepage
\Defgeneric bin2gray {translation-instance gray-coded-value}
\defmethod bin2gray {(translation-instance \inxclass{gray-code-translation}) value}

This function uses \arg{translation-instance} to convert the binary coded
\arg{value} to its gray coded equivalent, which it returns.
\par}%end samepage

\gap

\filbreak
{\samepage
The following example illustrates the use of these functions.\footnote{The code for
this example is included in a comment in the \cl{chromosome-methods.lisp} file.}
\begin{clcode}
(let ((gct (make-instance 'gray-code-translation
             :number-of-bits 5)))
  (format t "~&Int ~7TBinary ~19TGray ~23TGrayInt  RecoveredInt")
  (dotimes (i (expt 2 (number-of-bits gct)))
    (let ((g (bin2gray gct i)))
      (format t "~%~3D  ~8B  ~8B  ~4D  ~8D"
              i i g g (gray2bin gct g))))
  (format t "~2%GrayInt Int")
  (dotimes (i (expt 2 (number-of-bits gct)))
    (format t "~% ~6D ~3D" i (gray2bin gct i))))
\end{clcode}
}%end samepage


\filbreak
{\samepage
\subsubsection{Sequence Chromosomes} \label{sec:sequence-chromosomes}

\Defclassv {sequence-chromosome} {chromosome}

Sequence chromosomes are a subclass of \inxclass{chromosome} whose alleles are always
chosen such that every locus of a chromosome has an allele which does not occur at any
other locus of the chromosome. This requires that several operations which manipulate
these chromosomes be handled differently in order to maintain this property of uniqueness
of alleles within the chromosome.
\par}%end samepage

\filbreak
Note that \inxclass{sequence-chromosome} is still an abstract (non-instantiable) class,
since the size of the chromosome and the number of alleles (usually, but not necessarily
the same) are left unspecified.

\gap\filbreak

This class has no additional slots beyond those defined for the
\inxclass{chromosome} class.

\gap\filbreak
{\samepage


\bold{Instance Creation and Initialization}

The generic function \inxgeneric{make-chromosome} (see
page~\pageref{method:make-chromosome}) is the \geco\ interface for creation
of \inxclass{chromosome} instances.
\par}%end samepage

\gap

\filbreak
{\samepage


\bold{Specialized Methods}

\Defgeneric pick-random-alleles {chromosome}
\defmethod pick-random-alleles {(chromosome \inxclass{sequence-chromosome})}

This function initializes the loci of \arg{chromosome} to random alleles. The
\geco-supplied primary method assigns \term{allele codes} to each locus in
\arg{chromosome} corresponding to the locus' index into the \term{loci vector}, and the
calls \inxgeneric{scramble-alleles} on \arg{chromosome}.
The value returned by this function is not defined.
\par}%end samepage

\filbreak


{\samepage
\subsubsection{Sequence Genetic Operators}\index{genetic operators!sequence chromosomes}

\Defgeneric pmx-cross-chromosomes {\hbox{parent-1 parent-2}
                               \hbox{child-1 child-2}
                               \key :allele-test :locus-index1 :locus-index2}
\defmethod pmx-cross-chromosomes {\hbox{(parent-1 \inxclass{sequence-chromosome})}
                              \hbox{(parent-2 \inxclass{sequence-chromosome})}
                              \hbox{(child-1 \inxclass{sequence-chromosome})}
                              \hbox{(child-2 \inxclass{sequence-chromosome})}
    \key \hbox{(:allele-test \cl{\#'eql})}
         \hbox{(:locus-index1 \cl{(\inxfun{pick-random-locus-index} \arg{parent-1})})}
         \hbox{(:locus-index2 \cl{(\inxfun{pick-random-locus-index} \arg{parent-1})})}}

This function performs a partially mapped \term{crossover}
\cite{ga:goldberg} between the two parent chromosomes \arg{parent-1} and
\arg{parent-2}, storing the result in the two child chromosomes \arg{child-1} and
\arg{child-2}. The two arguments \arg{:locus-index1} and \arg{locus-index2} specify
the boundaries of the segment of \arg{parent-1} which is to be crossed with
\arg{parent-2}, defaulting as shown above. The \arg{:allele-test} argument specifies
a predicate to determine equality of two alleles, defaulting as shown above. The
\geco-supplied primary method treats the chromosome as circular when
\arg{:locus-index1} $>$ \arg{:locus-index2}. If \arg{:locus-index1} $=$
\arg{:locus-index2}, or if one is 0 and the other $=$ the length of the
parent chromosomes, then the children are simply copies of the parents.
The value returned by this function is not defined.
\par}%end samepage

\filbreak
{\samepage
\Defgeneric r3-cross-chromosomes {\hbox{parent-1 parent-2} \hbox{child-1 child-2}
                                  \key :allele-test}
\defmethod r3-cross-chromosomes {\hbox{(parent-1 \inxclass{sequence-chromosome})}
                                 \hbox{(parent-2 \inxclass{sequence-chromosome})}
                                 \hbox{(child-1 \inxclass{sequence-chromosome})}
                                 \hbox{(child-2 \inxclass{sequence-chromosome})}
                                 \key \hbox{(:allele-test \cl{\#'eql})}}

This function performs a random respectful recombination \term{crossover}
\cite{ga:radcliffe92-11,ga:radcliffe92-04}) between the two parent chromosomes
\arg{parent-1} and \arg{parent-2}, storing the result in the two child chromosomes
\arg{child-1} and \arg{child-2}. The \arg{:allele-test} argument specifies a
predicate to determine equality of two alleles, defaulting as shown above.
The value returned by this function is not defined.
\par}%end samepage
\filbreak


\section{The Genetic Plan Class} \label{sec:genetic-plan}

A \term{genetic plan} controls the overall strategy which
determines how an ecosystem \term{regenerate}s\index{regenerate}, \ie, how new
organisms are created from older organisms. This generally includes the overall
scheme for selection\index{selection} of organisms for reproduction and application
of \term{genetic operator}s. The actual selection methods\index{selection} provided by
\geco\ are described in Section~\ref{sec:selection-methods}, since they are
typically not specialized on the class of the \term{genetic plan}.

\gap

\filbreak
{\samepage
	\Defclass {genetic-plan}
	
	\gap
	
	\bold{Instance Allocated Slots}
	
	\Defslot {ecosystem}
	\definitarg {:ecosystem}
	\defaccessor {ecosystem}
	
	This slot records the \term{ecosystem} which is using the \term{genetic plan}.
	\par}%end samepage

\filbreak
{\samepage
	\Defslotv {generation-limit} {nil}
	\definitarg {:generation-limit}
	\defaccessor {generation-limit}
	
	\Defslotv {evaluation-limit} {nil}
	\definitarg {:evaluation-limit}
	\defaccessor {evaluation-limit}
	
	These slots (which default to \cl{nil}) can be used to establish
	termination\index{termination} criteria for the evolutionary process. They are used by
	the \geco-supplied primary method for \inxgeneric{evolution-termination-p} (see 
	page~\pageref{evolution-termination-p}).	
	\par}%end samepage
	
	\gap\filbreak
	
	{\samepage
		\bold{Instance Creation and Initialization}
		
		The generic function \inxgeneric{make-genetic-plan} (see page~\pageref{method:make-genetic-plan})
		is the \geco\ interface for creation of \inxclass{genetic-plan} instances.
	\par}%end samepage
	
	\gap
	
	\filbreak
	{\samepage
		
		\bold{Specialized Methods}
		
		\Defgeneric regenerate {plan thing}
		\defmethod regenerate {(plan \inxclass{genetic-plan}) (ecosystem \inxclass{ecosystem})}
		\defmethod regenerate {(plan \inxclass{genetic-plan}) (old-population \inxclass{generational-population})}
		\label{method:regenerate}
		
		This function creates a new version of \arg{thing} which is more evolved according
		to the \term{genetic plan} \arg{plan}. The \geco-supplied version of \inxmethod{regenerate} which
		is specialized to the class \inxclass{ecosystem} invokes \inxfun{regenerate} on
		\arg{ecosystem}'s population, and saves the result in \arg{ecosystem}'s
		\inxslot{population} slot.
		\par}%end samepage
	
	\gap
	
	\filbreak
	Note that \inxclass{generational-population} is currently the only
	population class for which \hbox{\inxgeneric{regenerate}} is defined.
	
	\filbreak
	The \geco-supplied version of \inxmethod{regenerate} which is specialized to the class
	\inxclass{generational-population} is not intended to be used for real GAs, but to serve
	as a template to illustrate the responsibilities of \inxgeneric{regenerate}. Therefore a
	specialized method should be implemented for all subclasses of \inxclass{population},
	including \inxclass{generational-population}.\footnote{There are comments at the
		beginning of the {\tt generics.lisp} file which summarize the functions which should or
		must be defined to implement a working GA using \geco.} For generational
	GAs, the responsibilities of \inxgeneric{regenerate} include:
	\filbreak
	{\samepage
		\begin{itemize}
			
			\item Create a new population of the same class as
			\arg{old-population}, and whose size is {\em based on} the size
			of \arg{old-population}.  
			Note that the new population need not necessarily be the same
			size as \arg{old-population} unless that is consistent with the
			\term{genetic plan}. Note also that this size is the size of the
			\inxslot{organisms} vector.
			
			\item Assure that the \inxslot{ecosystem} slot of the new population is the
			same as that of \arg{old-population}.
			
			\item Install organisms in the \term{organisms vector} of the new population, 
			based on the organisms of \arg{old-population}.  This typically involves:
			\begin{itemize}
				\item Selecting\index{selection} some of the organisms from
				\arg{old-population} to participate in creation of the new
				population.  This selection process is typically based on
				their \concept{score}s (fitness or penalty), and may be performed using
				one or more selection methods (see Section~\ref{sec:selection-methods}), 
				or similar methods.
				\item Copying some of the selected organisms from \arg{old-population}, and
				\item Creating new organisms to include in the new population.
				Some organism may simply be copied to the new population, but many new
				organisms are created by applying \term{genetic operator}s, such as
				\term{mutate} or \term{crossover} on members of \arg{old-population}.
			\end{itemize}
			\item Returns the new \inxclass{population} instance.
		\end{itemize}
	}%end samepage
	
	\filbreak
	{\samepage
		\Defgeneric evolution-termination-p {plan}  \label{evolution-termination-p}
		\defmethod evolution-termination-p {(plan \inxclass{genetic-plan})}
		
		This function is a predicate used by the \geco-supplied method \inxmethod{evolve} to
		determine when to terminate\index{termination} the evolutionary process. The
		\geco-supplied primary method returns true (non-\cl{nil}) when either an evaluation
		limit or a generation limit has been established (by putting a number in the
		\inxslot{evaluation-limit} or the \inxslot{generation-limit} slot of \arg{plan}) and
		either of those limits has been exceeded, or when \inxgeneric{converged-p}
		(page~\pageref{population:converged-p}) returns true.
		\par}%end samepage
	\filbreak
	
	\section{Selection Methods}
	\label{sec:selection-methods}
	
	\Geco\ provides a sampling of selection\index{selection} methods. None of them are
	guaranteed to be the best in the world, but some of them may prove useful as examples, or
	as a base upon which to build your own.
	
	\filbreak
	{\samepage
		\Defgeneric pick-random-organism-index {population}
		\defmethod pick-random-organism-index {(population \inxclass{population})}
		
		This function returns the index of a random organism from \arg{population}. The
		\geco-supplied primary method simply calls \inxfun{geco-random-integer} with the
		argument \cl{(size \arg{population})}.
		\par}%end samepage
	
	\filbreak
	{\samepage
		\Defgeneric pick-random-organism {population}
		\defmethod pick-random-organism {(population \inxclass{population})}
		
		This function returns a random organism from \arg{population}.  The
		\geco-supplied primary method returns the organism from \arg{population}
		indexed by the value returned from \inxgeneric{pick-random-organism-index}.
		\par}%end samepage
	
	\filbreak
	{\samepage
		\Defun roulette-pick-random-weight-index {weights-table \key :invert-p}
		
		This function selects and returns a random index into an array of weights \arg{weights-table}, using
		the roulette wheel approach \cite{ga:goldberg}. An entry in \arg{weights-table}
		indicates the probability that the corresponding index should be returned. The
		\arg{:invert-p} argument when non-\cl{nil} causes the selection to be inversely
		proportional to \arg{weights-table} entries. The \geco-supplied primary method assumes
		that \arg{weights-table} has been normalized to sum to $1.0$.
		\par}%end samepage
	
	\filbreak
	{\samepage
		\Defmethod roulette-pick-random-organism-index {population}
		\defgeneric roulette-pick-random-organism-index {(population \inxclass{population})}
		
		This function selects and returns a random organism index from \arg{population}, weighted by
		\term{score}, using the roulette wheel approach \cite{ga:goldberg}, as used in
		DeJong's R1 \cite{ga:dejong-thesis}; it is also referred to by Brindle as stochastic
		sampling with replacement \cite{ga:brindle}.
		\par}%end samepage
	
	\filbreak
	{\samepage
		\Defmethod roulette-pick-random-organism {population}
		\defgeneric roulette-pick-random-organism {(population \inxclass{population})}
		
		This function selects and returns a random organism from \arg{population}, weighted by
		\term{score}, using the roulette wheel approach \cite{ga:goldberg}, as used in
		DeJong's R1 \cite{ga:dejong-thesis}; also referred to by Brindle as stochastic
		sampling with replacement \cite{ga:brindle}.
		\par}%end samepage
	
	\filbreak
	{\samepage
		\Defmethod stochastic-remainder-preselect {population \key :multiplier}
		\defgeneric stochastic-remainder-preselect {(population \inxclass{population})
			\key (:multiplier \cl{1})}
		\label{population:stochastic-remainder-preselect}
		This function prepares and returns a function (actually a closure) of no arguments which will
		select and return random organisms from \arg{population}, weighted by \term{score}, using a
		technique referred to by Brindle as stochastic remainder selection without replacement
		\cite{ga:brindle}. Each call to the returned function will return an organism member of
		\arg{population} until the appropriate number of organisms have been selected, then the
		function will return \cl{nil}. The \arg{:multiplier} keyword argument can be supplied to
		indicate the number of organisms to be selected, in terms of the size of \arg{population}.
		For instance, if it is desired that the returned function return twice as many organisms as
		are in \arg{population}, a \arg{:multiplier} value of $2$ should be used.
		\par}%end samepage
	
	\filbreak
	{\samepage
		The following code fragment illustrates the intended use:
		\begin{clcode}(let ((selector (stochastic-remainder-preselect some-population)))
  (do ((organism (funcall selector) (funcall selector)))
      ((null organism))
    (do-something-with organism)))\end{clcode}
	}%end samepage
	
	\filbreak
	
	{\samepage
		\Defmethod ranking-preselect {population \key :multiplier :max}
		\label{method:ranking-preselect}
		\defgeneric ranking-preselect {(population \inxclass{population})
			\key (:multiplier \cl{1}) (:max \cl{2.0})}
		
		This function prepares and returns a function (actually a closure) of no arguments which will
		select and return random organisms from \arg{population}, weighted by the rank of each
		organism's \term{score} wthin \arg{population}, without replacement.
		\cite{ga:baker-rank-selection} Each call to the function returned from this method will
		return an organism member of \arg{population} until the appropriate number of organisms have
		been selected, then the function will return \cl{nil}. The \arg{:multiplier} keyword argument
		can be supplied to indicate the number of organisms to be selected, in terms of the size of
		\arg{population}. For instance, if it is desired that the returned function return twice as
		many organisms as are in \arg{population}, a \arg{:multiplier} value of $2$ should be used.
	\par}%end samepage

	\filbreak
	
	{\samepage
		The main idea of rank selection (as implemented here) is as follows: Sort the population by
		score from best to worst, assigning a linearly decreasing number of copies to each organism,
		starting with \arg{:max} copies of the most fit organism. The number of copies of the least
		fit organism is determined according to the following formula:
		
		%   $$\arg{:max} - 2.0 (\arg{:max} - 1.0)$$
		%   $$\mathord{\arg{:max}} - 2.0 (\mathord{\arg{:max}} - 1.0)$$
		
		\def\KWmaxArg{\mbox{\arg{:max}}}%
		$$\mathord{\KWmaxArg} - 2.0\ (\mathord{\KWmaxArg} - 1.0)$$
	}%end samepage

\filbreak

{\samepage
		where fractional remainders are used as probabilities, and negative values are equivalent to
		zero. Note that values for \arg{:max} greater than $2.0$ will result in some fraction of the
		less fit organisms in \arg{population} not being selected at all.
		
		\par}%end samepage
	
	\filbreak
	
	{\samepage
		\Defgeneric pick-some-random-organism-indices {population number-to-pick}
		\defmethod pick-some-random-organism-indices {(population \inxclass{population}) number-to-pick}
		
		This function returns \arg{number-to-pick} random organism indices for \arg{population}.
		The indices will each be unique, \ie, there will be no duplicates for any given call to
		this function.
		\par}%end samepage
	
	\filbreak
	{\samepage
		\Defgeneric tournament-select-organism {population tournament-size}
		\defmethod tournament-select-organism {(population \inxclass{population}) tournament-size}
		
		This function picks \arg{tournament-size} organisms from \arg{population} at random, and
		returns the best (most fit) one. The \geco-supplied method calls
		\inxgeneric{pick-some-random-organism-indices} to establish the members of the
		tournament, and uses \inxgeneric{better-than-test} to compare the organisms.
		\par}%end samepage
	\filbreak
	
	
	\section{The Population Statistics Class} \label{sec:pop-stats-class}
	
	This class supports accumulation of information about (at least) the \term{scores}
	of the members of a population. This information can be used for
	normalizing\index{normalization} the scores across the population, \etc.
	
	Instances are created automatically by \geco{} at the end of evaluating
	a new population, after all the organisms have been created and evaluated.
	
	\filbreak
	{\samepage
		\Defclass {population-statistics}
		
		\gap
		
		\bold{Instance Allocated Slots}
		
		\Defslot {population}
		\definitarg {:population}
		\defaccessor {population}
		
		This slot indicates the population to which this population-statistics
		instance applies.  The \cl{:population} initarg should be specified when
		a \inxclass{population-statistics} instance is created.
		\par}%end samepage
	
	\filbreak
	{\samepage
		\Defslot {sum-score}
		\definitarg {:sum-score}
		\defaccessor {sum-score}
		\par}%end samepage
	
	\filbreak
	{\samepage
		\Defslot {avg-score}
		\definitarg {:avg-score}
		\defaccessor {avg-score}
		\par}%end samepage
	
	\filbreak
	{\samepage
		\Defslot {max-score}
		\definitarg {:max-score}
		\defaccessor {max-score}
		\par}%end samepage
	
	\filbreak
	{\samepage
		\Defslot {min-score}
		\definitarg {:min-score}
		\defaccessor {min-score}
		\par}%end samepage
	
	\filbreak
	{\samepage
		\Defslot {max-organism}
		\definitarg {:max-organism}
		\defaccessor {max-organism}
		\par}%end samepage
	
	\filbreak
	{\samepage
		\Defslot {min-organism}
		\definitarg {:min-organism}
		\defaccessor {min-organism}
		\par}%end samepage
	
	\medskip
	
	
	
	\filbreak
	{\samepage
		These slots hold the calculated values, respectively, for:
		\begin{itemize}
			\item the sum of the \inxslot{score}s of all the organisms in the \inxslot{population}
			\item the average (statistical mean) of the \inxslot{score}s of all the
			organisms in the \inxslot{population}
			\item the maximum of the \inxslot{score}s of all the organisms in the \inxslot{population}
			\item the minimum of the \inxslot{score}s of all the organisms in the \inxslot{population}
			\item an organism in \inxslot{population} which had a score of \inxslot{max-score}
			\item an organism in \inxslot{population} which had a score of \inxslot{min-score}
		\end{itemize}
	}%end samepage
	
	The above values are calculated by \inxgeneric{compute-statistics},
	which is invoked automatically at the end of initialization
	of an instance of a \inxclass{population-statistics} class.
	
	\filbreak
	{\samepage
		\Defslot {sum-normalized-score}
		\definitarg {:sum-normalized-score}
		\defaccessor {sum-normalized-score}
		
		\Defslot {avg-normalized-score}
		\definitarg {:avg-normalized-score}
		\defaccessor {avg-normalized-score}
		
		These slots hold the calculated values, respectively, for:
		\begin{itemize}
			\item the sum of the normalized\index{normalization} \inxslot{score}s of all the
			organisms in the \inxslot{population}
			\item the average (statistical mean) of the normalized \inxslot{score}s of all the
			organisms in the \inxslot{population}
		\end{itemize}
	}%end samepage
	
	The above values are calculated by \inxgeneric{compute-normalized-statistics}, which
	\geco{} invokes automatically as part of evaluatiing a population (see
	figure~\ref{fig:initialization-tree}, page~\pageref{fig:initialization-tree}).
	
	\gap
	
	\filbreak
	
	{\samepage
		\bold{Instance Creation and Initialization}
		
		The generic function \inxgeneric{make-population-statistics} (see
		page~\pageref{method:make-population-statistics}) is the \geco\ interface for creation of
		\inxclass{population-statistics} instances.
	\par}%end samepage
		
	\gap
	
	\filbreak

	{\samepage

		\bold{Specialized Methods}
	
		% use \Eggeneric so the CLOS generic isn't indexed
		\Eggeneric shared-initialize {standard-object slot-names \rest initargs}
		\defaftermethod shared-initialize {(pop-stats \inxclass{population-statistics}) slot-names \rest initargs}
	
		This method extends the initialization for \inxclass{population-statistics} instances.
		In the \geco-supplied default method, if the \inxslot{sum-score} slot of 
		\arg{pop-stats} is unbound, the \inxgeneric{compute-statistics} generic function
		(see below) is invoked to calculate the non-normalized statistics slots of 
		\arg{pop-stats}.
		The slots initialized include: \inxslot{sum-score}, \inxslot{avg-score}, \inxslot{max-score},
		\inxslot{min-score}, \inxslot{max-organism}, and \inxslot{min-organism}.
	\par}% end \samepage
	
	\filbreak
	{\samepage
		% use \Eggeneric so the CLOS generic isn't indexed
		\Eggeneric print-object {standard-object stream}
		\defmethod print-object {(self \inxclass{population-statistics}) stream}
		
		This method specializes the standard Common Lisp \inxfun{print-object} generic function for
		instances of the \inxclass{population-statistics} class. It uses the standard Common Lisp
		function \inxfun{print-unreadable-object}, includes the type and identity of \arg{self}, and
		also includes one of the following:
		\begin{itemize}
			\item If the population is converged, the \inxslot{avg-score} of
			\arg{self}, which is the value to which all the organisms have
			converged, else
			\item Both the \inxslot{avg-score} and \inxslot{avg-normalized-score} of 
			\arg{self}.
		\end{itemize}
	\par}%end samepage
	
	\filbreak
	{\samepage
		\Defgeneric compute-statistics {population-statistics}
		\defmethod compute-statistics {(population-statistics
			\inxclass{population-statistics})}
		
		\label{compute-population-statistics}
		
		This function calculates and stores whatever statistics of the population are necessary for
		the \term{genetic plan} to calculate normalized scores of the organisms of the population indicated
		by the \inxslot{population} slot. The function is called by a \geco-supplied initialization
		method on the \inxclass{population-statistics} class. The \geco-supplied primary method
		calculates the sum of all the scores of the organisms in the \inxslot{population}, and the
		minimum, maximum, and average scores for the \inxslot{population}, and retains (pointers to)
		organisms in \inxslot{population} which have the minimum and maximum scores. These values are
		stored in the appropriate slots of \arg{population-statistics}.
		The value returned by this function is not defined.
		\par}%end samepage
	
	\filbreak
	{\samepage
		\Defgeneric compute-normalized-statistics {population-statistics}
		\defmethod compute-normalized-statistics {(population-statistics \inxclass{population-statistics})}
		
		This function calculates and stores whatever statistics of the
		normalized\index{normalization} scores of the \inxslot{population} are necessary for
		the \term{genetic plan} to control the evolution of the ecosystem at
		the current time. The function is automatically called by a \geco-supplied
		\inxmethod{normalize-score} method which is specialized to the
		\inxclass{population}, \inxclass{population-statistics}, and \inxclass{genetic-plan}
		classes. The \geco-supplied primary method calculates the sum of all the normalized
		scores of the organisms in the \inxslot{population}, and the average normalized
		score for the \inxslot{population}. These values are stored in the appropriate slots
		of \arg{population-statistics}.
		The value returned by this function is not defined.
		\par}%end samepage
	\filbreak
