\chapter{The GECO Files}\label{ch:geco-files}

This section provides a brief overview of the \geco\ files.  The files are 
discussed in groups, based on related type or content.

The first group of files provide documentation.
\begin{itemize}

  \item \path|README|\\
  An overview of the \geco\ distribution, including abstract, copyright and 
  authorship information, installation instructions, version history.

  \item \path|COPYING.LIB-2.0|\\
  A copy of the GNU Library General Public License, version 2.0, which describes the 
  terms under which \geco\ is distributed.
  This document is a product of the Free Software Foundation, Inc., of 
  Cambridge, Mass.

  \item \path|geco.pdf|\\
  A copy of the \geco\ documentation (this document), in Portable Document Format form.

\end{itemize}

The next file describes the structure and files provided with the \geco\ system for purposes
of compiling and loading it into a lisp environment.

\begin{itemize}

\item \path|geco.asd|\\ 
The system definition file. It contains the \cl{defsystem} form for creating \geco\ using {\sc asdf}.
{\sc Asdf}, which stands for {\em Another System Definition Facility} is the most common
mechanism for specifying system definitions for Common Lisp systems.
Please see \href{https://common-lisp.net/project/asdf/}{https://common-lisp.net/project/asdf/}
for information on how to install and use {\sc asdf}.

It is also worth noting here that some of the \geco\ source code files are in
subdirectories to facilitate the specification of {\em modules} in the {\sc asdf} \cl{defsystem}
form in this file. These {\em modules} facilitate the expression of dependencies among the files.
Where files are in subdirectories, relative path-names have been used below.


\end{itemize}

The next group of files contain definitions which must be loaded/compiled 
before the rest of the \geco\ source code. Such dependencies and ordering is 
handled automatically by {\sc asdf} using the \path|geco.asd| file.
\begin{itemize}
	
	\item \path|packages.lisp|\label{file:packages.lisp}\\
	This file defines the \geco\ packages, of which there are two:
	\begin{enumerate}
		\item \cl{GECO}\\
		This package contains all of the code that implements \geco.
		Most of the definitions of classes, functions, methods, etc., are exported from this package.
		
		\item \cl{GECO-USER}\\
		This is the package that is typically used for user-defined code that uses \geco. It \cl{use}s
		the \cl{GECO} package.
	\end{enumerate}
	
	In addition, this file has a small amount of code that uses some conditional compilation features
	to configure \geco\ to use one of two
	random number generators, and to handle implementations that do not support tail-recursion optimization.
	You should look at this code to determine if you need to change it for your system.	

  \item \path|definitions/generics.lisp|\\
  This file contains \cl{defgeneric} forms defining some (but not all)
  of the generic function 
  {\em protocol} established by \geco, \ie, the set of generic
  functions and their arguments which must be honored by all \geco-based
  applications. Each of the \cl{defgeneric} forms contains a
  \cl{:documentation} string for the function describing its intended purpose
  (these documentation strings are easily retrieved in most interactive lisp
  programming environments). Comments in the file also indicate which of the
  generic functions should/must have methods defined for your
  application-specific classes when you implement a GA with \geco.

  \item \path|definitions/classes.lisp|\\
  This file contains the \cl{defclass} forms defining each of the \geco\ 
  classes.

\filbreak
  \item \path|utilities/dbg.lisp|\\
  This file contains the definitions for a simple debugging facility used 
  in the development of \geco, and which may be useful for instrumenting user-developed
  code as well. It should be obvious how it works, so you might want to take a look.

\filbreak
  \item \path|utilities/random.lisp|\\
  This file contains the definition of the random number generators used by \geco,
  \inxfun{geco-random-integer} and \inxfun{geco-random-float}.  In addition, it
  includes the definition of an alternate set of random number generators, provided
  with permission from John Koza from his implementation Simple Genetic
  Programming in Lisp. Conditional compilation options (in \path|packages.lisp|, as mentioned above)
  control which random number generator is used.

\end{itemize}

Most of the remaining files contain the method definitions for the guts of 
\geco.  An attempt has been made to organize them by the principle class 
to which the methods apply, however, due to the use of multiple-dispatch 
methods, this has not always been possible.

\filbreak
In general, the files have been named using a standard pattern:
{\it class-name}\path|-methods.lisp|. Presently, the single exception is 
the file \path|selection-methods.lisp|, which I decided to separate from 
the other \inxclass{population} methods.

\filbreak

\begin{itemize}

  \item \path|methods/ecosystem-methods.lisp|\\
  This file contains methods which perform the following general categories 
  of operations:
	\begin{itemize}
	 \item initialize\index{initialization} ecosystems
	 \item make instances of \inxclass{population} and \inxclass{genetic-plan} 
		appropriate for an ecosystem
	 \item evolve and evaluate ecosystems
	\end{itemize}

  \item \path|methods/genetic-plan-methods.lisp|\\
  This file contains methods which perform the following general categories 
  of operations:
	\begin{itemize}
	 \item regenerate instances of \inxclass{ecosystem} and \inxclass{population}
	 \item determine when evolution should be terminated\index{termination}
	\end{itemize}

\filbreak
  \item \path|methods/population-methods.lisp|\\
  This file contains methods which perform the following general categories 
  of operations:
	\begin{itemize}
	 \item initialize\index{initialization} and print instances of \inxclass{population}
	 \item create \inxclass{organism} and \inxclass{population-statistics}
               instances for a population 
	 \item evaluate populations, and compute statistics over them
	 \item compute normalized\index{normalization} scores over populations
	 \item determine if a population has converged
	\end{itemize}

\filbreak
  \item \path|methods/pop-stats-methods.lisp|\\
  This file contains methods which perform the following general categories 
  of operations:
	\begin{itemize}
	 \item initialize, and print instances of \cl{population-statistics}
	 \item compute and normalize\index{normalization} population statistics
	\end{itemize}

\filbreak
  \item \path|methods/selection-methods.lisp|\\
  A fairly broad sampling of techniques for selecting \index{selection}
  organisms from populations.  Techniques include:
	\begin{itemize}
	 \item random selection
	 \item weighted roulette-wheel selection
	 \item stochastic remainder selection
	 \item tournament selection
	\end{itemize}
  A version of the roulette-wheel selection routine has also been generalized
  to select an index from a table of weights. I expect this to be
  useful for performing weighted \term{genetic operator} selection.

\filbreak
  \item \path|methods/organism-methods.lisp|\\
  This file contains methods which perform the following general categories 
  of operations:
	\begin{itemize}
	 \item initialize\index{initialization}, copy, and print instances of \inxclass{organism}
	 \item create chromosomes for an organism
	 \item evaluate and decode organisms
	 \item compute normalized\index{normalization} scores of organisms
	 \item determine if two organisms are the same
	 \item choose random chromosomes, and random locations on chromosomes
	 \item apply \term{genetic operator}s, such as \term{mutate} or \term{crossover} on organisms
	\end{itemize}

\filbreak
  \item \path|methods/chromosome-methods.lisp|\\
  This file contains methods which perform the following general categories 
  of operations:
	\begin{itemize}
	 \item initialize\index{initialization}, copy, and print instances of \inxclass{chromosome}
	 \item create and print \term{loci vectors}
	 \item access individual loci
	 \item pick random loci and allele values
	 \item count allele values
	 \item convert (internal) \term{allele codes} to (printable)
               \term{allele values} 
	 \item decode binary chromosomes
	 \item gray-coded representations
	 \item determine if two chromosomes are the same
	 \item determine the Hamming distance between two chromosomes
	 \item \term{genetic operator}s, such as \term{mutate} or \term{crossover} on organisms
	\end{itemize}

\end{itemize}

Finally, there are also some files containing example GAs and related code.
These files aren't intended to show impressive solutions to tough problems;
rather they are intended to show how one might go about building a GA using \geco.

\begin{itemize}
	\item \path|sb-test.lisp|\\
	This is the simple binary example discussed in
	Chapter~\ref{chap:simple-binary-example}.

	\item \path|allele-counts.lisp|\\
	This file provides some utilities for counting alleles. It is fairly general, but has
	not been tested enough yet to be included in \geco-proper.
	
	This file is required by \path|sb-test.lisp|

  \item \path|ss-test.lisp|\\
  This is the simple sequence example discussed in
  Chapter~\ref{chap:simple-sequence-example}
\end{itemize}


