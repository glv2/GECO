\chapter{A Simple Sequence Example} \label{chap:simple-sequence-example}

An example of how to customize \geco\ is provided in the file \path|ss-test.lisp|,
which presents an example GA to solve a simple problem for optimizing a sequence of things.
Sequence optimization is often used to solve problems such as the Traveling Salesman
Problem.\footnote{See \href{https://en.wikipedia.org/wiki/Travelling\_salesman\_problem}{https://en.wikipedia.org/wiki/Travelling\_salesman\_problem}}
The following material provides an overview of the definitions in this file,
providing some discussion of each, why it is necessary and/or
what it does, and how it fits into the \geco\ framework.

\filbreak

\section{Using \Geco\ with Packages}

See section \ref{sec:using-geco-packages}.

\filbreak

\section{Defining the Genetic Structures}

{\samepage
The file starts with the definition of some special (dynamically bound) variables,
most of which are used to simplify testing.
\begin{clcode}(defvar *SSE* nil {\sl "a simple sequence ecosystem"})
(defvar *SSE-STATISTICS-FILE-COUNTER* 0)
(defvar *SSE-STATISTICS-FILE-NAME* "sse-stats")
(defvar *SSE-STATISTICS-STREAM*) ; {\sl should probably be a slot on ecosystem}
(defvar *SSE-DEFAULT-POP-SIZE* 40)
(defvar *SSE-DEFAULT-EVAL-LIMIT* 3000)
(defvar *SSE-PROB-MUTATE* 0.025)
(defvar *SSE-PROB-CROSS* 0.8)\end{clcode}
}%end \samepage

\filbreak

{\samepage

Next, there is a \cl{let*}-form which allows generalizing the definition of the
\term{chromosome} class and some of its methods to simplify changing the sequence length.
This form could be used to generalize many of the other definitions in the file, but that
is left as an exercise... The effect of this \cl{let*} form is to create the following
definitions:
\begin{clcode}(defclass SSE-CHROMOSOME-10 (sequence-chromosome)
  ())

(defmethod SIZE ((self sse-chromosome-10))
  10))

(defmethod LOCUS-ARITY ((self sse-chromosome-10) locus-index)
  (declare (ignore locus-index))
  10)

(defmethod PRINTABLE-ALLELE-VALUES ((self sse-chromosome-10) locus-index)
  (declare (ignore locus-index))
  #(#\\0 #\\1 #\\2 #\\3 #\\4 #\\5 #\\6 #\\7 #\\8 #\\9))\end{clcode}
}%end \samepage

\filbreak

Our new class is a specialized version of \inxclass{sequence-chromosome}
(see Section \ref{sec:sequence-chromosomes}), and the above method definitions proviide
\geco\ with information it needs to implement some of its functionality, just as
we did in the previous chapter's example.

Next is the definition of a specialized form of crossover that works on \inxclass{sse-chromosome-10}
instances.
This method is actually a slightly modified version of the \inxmethod{r3-cross-organisms} that is
built into \geco\ for all instances of the \inxclass{sequence-chromosome} class (and its subclasses),
but the method here illustrates how the standard \geco-supplied method can be copied and
modified. This method adds a heuristic aspect to the original method (which is not necessarily better).


{\samepage
Next, we define \inxclass{sse-10-organism} as a subclass of \inxclass{organism}.
This class will hold a single chromosome of the class we just defined in its
\inxslot{genotype} slot. And using the same technique as in the previous paragraph, we define
a method \inxmethod{chromosome-classes} for this class to give \geco\ a list of classes
for the chromosomes which will be held by instances of this subclass of organism.
Specifically, this method returns a list of length one (since we only need one
chromosome), and the sole list element is the name of our application specific
chromosome class, \inxclass{sse-chromosome-10}.
\begin{clcode}(defclass SSE-ORGANISM-10 (organism)
  ())

(defmethod CHROMOSOME-CLASSES ((self sse-organism-10))
  '(sse-chromosome-10))\end{clcode}
}%end \samepage

\filbreak

{\samepage
The next class definition is a specialization of the \geco\ class
\inxclass{population-statistics}. Instances of this class are used by \geco\ to record
statistical information about the current population to simplify certain operations
like normalizing\index{normalization} scores and determining whether the population
has \term{converged}. By specializing this class, we'll be able to
piggy-back some of the calculations we want performed on the functions which \geco\
will already be invoking. To record the additional information we want, we add a couple
of slots to the class.
\begin{clcode}(defclass SSE-POPULATION-STATISTICS-10 (population-statistics)
  (;; {\sf the following slots are set by class specific compute-statistics{\tt :after} method }
   (SUM-SQUARES
    :accessor sum-squares
    :initarg :sum-squares
    :type number)
   (STD-DEVIATION
    :accessor std-deviation
    :initarg :std-deviation
    :type number)))\end{clcode}
}%end \samepage

\filbreak

{\samepage
To fill in these additional slots, we define a shorthand function \cl{SQR}, and an \cl{:after}
method on \geco's builtin \inxmethod{compute-statistics} method:
\begin{code}(defun SQR (n)
  (* n n))

(defmethod COMPUTE-STATISTICS :AFTER ((self sse-population-statistics-10))
  (with-accessors ((size size)
                   (orgs organisms))
                  (population self)
    (with-accessors ((sum sum-score)
                     (sum-sq sum-squares)
                     (stdev std-deviation))
                    self
      (setq sum-sq 0.0)
      (dotimes (i size)
        (incf sum-sq (sqr (score (aref orgs i)))))
      (setq stdev (sqrt (/ (- sum-sq (float (/ (sqr sum) size)))
                           (1- size)))))))\end{code}
}%end \samepage

\filbreak

{\samepage

We then also provide a method to pretty-print objects of this new class:
\begin{code}(defmethod PRINT-OBJECT ((self sse-population-statistics-10) stream &AUX
                         (sse (ecosystem (population self))))
  (print-unreadable-object (self stream :type t :identity t)
    (if (slot-boundp self 'max-score)
        (format stream "Max=~F, Avg=~,2F, StdDev=~,2F, #Gens=~D, #Evals=~D"
                (max-score self)
                (avg-score self)
                (std-deviation self)
                (generation-number sse)
                (evaluation-number sse))
      (princ "#unbound#" stream))))\end{code}

}%end \samepage

\filbreak

The next class definition is a specialization of the class
\inxclass{generational-population}, which is itself a specialization of the class
\inxclass{population}. We also add the \inxclass{maximizing-score-mixin} so that
\inxmethod{converged-p} will have the information it needs to determine whether the
population has \term{converged}.

\filbreak

{\samepage

The principle noteworthy feature of this subclass,
\inxclass{sse-population-10}, is that it provides some additional methods to
provide information to \geco\ so that it can perform its duties
automatically. Specifically, the \inxmethod{organism-class} method tells \geco\ what
class the organism instances are to be, and the
\inxmethod{population-statistics-class} method tells \geco\ what class the population
statistics instances are to be, while the other methods provide
\begin{clcode}(defclass SSE-POPULATION-10 (generational-population maximizing-score-mixin)
  ())

(defmethod ORGANISM-CLASS ((self sse-population-10))
  'sse-organism-10)

(defmethod POPULATION-STATISTICS-CLASS ((self sse-population-10))
  'sse-population-statistics-10)\end{clcode}

}%end \samepage

\filbreak

{\samepage

At this point please notice that telling \geco\ to create an instance of the class
\inxclass{sse-population-10} is sufficient, and that \geco\ can then create a
complete population of organisms of the proper class, and that each organism will
contain chromosomes of the proper class, and each chromosome will have a \term{loci
vector} of the proper size and type, initialized\index{initialization} to random
alleles that are unique to the chromosome; \ie, a sequence, albiet not necessarily in order.
Thus, the structures (at this level) which will be manipulated by our GA are
completely specified. Next, we need to specify the plan which controls
the GA.

}%end \samepage

\filbreak


\section{Defining a Genetic Plan}

{\samepage
	
Next we introduce a class definition that is a specialization of the class \inxclass{genetic-plan}.
As mentioned earlier, the \term{genetic plan} provides a strategy which determines how an ecosystem
regenerates, \ie, how new organisms are created from older organisms. This is the
heart of the genetic algorithm. Subclasses of the class \inxclass{genetic-plan} are
primarily used to specialize methods which perform the actual processing of the GA. To start with,
that includes some methods to supply probabilities for applying the \term{genetic operator}s.
\begin{clcode}(defclass SSE-PLAN (genetic-plan)
  ())

(defmethod PROB-MUTATE ((self sse-plan))
  *sse-prob-mutate*)

(defmethod PROB-CROSS ((self sse-plan))
  *sse-prob-cross*)\end{clcode}

}%end \samepage

\filbreak


Next is an \inxmethod{evaluate} method specialized on both the \inxclass{sse-plan}
and \inxclass{sse-organism-10} classes. This is the method which calculates the
raw (unnormalized) score of each organism which our plan evolves. In our specific
problem, score is the sum of the absolute values of the difference between a loci's
value and the value it should be.

\filbreak

{\samepage

Note that in most applications, this kind of knowledge
is unlikely to be available. If it were, why would we need a GA? In general,
\inxmethod{evaluate} algorithms will need to incorporate heuristics, and the
heuristics chosen by the developer will have a critical impact on guiding the
search.
\begin{clcode}(defmethod EVALUATE ((self sse-organism-10) (plan sse-plan) &AUX
                     (chromosome (first (genotype self)))
                     (chromosome-size (size chromosome)))
  (declare (ignore plan))
  (setf (score self)
  (do* ((locus# 0 (1+ locus#))
        (result 0))
       ((>= locus# chromosome-size)
        result)
    (incf result (- 10 (abs (- locus# (locus chromosome locus#))))))))\end{clcode}

}%end \samepage

\filbreak

{\samepage
As in the previous chapter, a few additional points worth noting about \inxgeneric{evaluate}:
\begin{itemize}
  \item This is the only place in our example GA which needs 
	to interpret the genetic content of our application-specific organism.
  \item Often it is necessary to \term{decode} the genetic content of an 
	organism, converting the \term{genotype} into an instance of the
	\term{phenotype} represented by the genotype.  \Geco\ provides for this by 
	including the following:
    \begin{itemize}
	\item A \inxslot{genotype} slot is defined in the \inxclass{organism} class.
	\item A \inxslot{phenotype} slot is defined in the
		\inxclass{organism-phenotype-mixin} class.
	\item A \inxgeneric{decode} generic function is called in a \cl{:before} method 
		of \inxgeneric{evaluate} specialized on the
       	\inxclass{organism-phenotype-mixin} and  
		\inxclass{genetic-plan} classes.
	\item Since \geco\ cannot predefine a method for \inxgeneric{decode}, any GAs
		using phenotypes must be sure to implement one for the application
		specific subclass of \inxclass{organism} and
		\inxclass{organism-phenotype-mixin}.
    \end{itemize}
  \item Generally there is little reason for the plan to be an argument to
	this particular method, but it is part of the protocol for the
	\inxgeneric{evaluate} generic function, which is also used at the 
	\inxclass{ecosystem} and \inxclass{population} levels of our class hierarchy, and
        at these higher levels it may well be appropriate for the
        \term{genetic plan} to discriminate between alternate methods.  
\end{itemize}
}%end samepage

\filbreak

{\samepage

The next method defined in our example is \inxmethod{regenerate}, which is specialized
on both the \inxclass{sse-plan} and the \inxclass{sse-population-10}
classes. The purpose of this method is to create a new (or revised) population based
on the current one, using whatever strategy is specific to the \term{genetic plan}.
There is a default method provided by \geco\ in
\path|genetic-plan-methods.lisp| for subclasses of \inxclass{generational-population},
but it is provided as a template, not a realistic example, since it simply copies
random organisms from one generation to the next (plus some simple bookkeeping). Our
specialized version of \inxmethod{regenerate} replaces the random copying with a call
to a new generic function \inxgeneric{operate-on-population} which takes the current
and new (but empty) populations as input, and updates the new population.
\begin{clcode}
(defmethod REGENERATE ((plan sse-plan)
                       (old-pop sse-population-10) &AUX
                       (new-pop (make-population (ecosystem old-pop)
                                                 (class-of old-pop)
                                                 :size (size old-pop))))
  ;; {\sf selectively reproduce, crossover, and mutate}
  (operate-on-population plan old-pop new-pop)
  ;; {\sf record old-pop's statistics}
  (print (statistics old-pop) *sse-statistics-stream*)
  new-pop)\end{clcode}
}%end \samepage

\filbreak

The method \inxmethod{operate-on-population} for \inxclass{sse-plan} uses a
selection technique referred to as ranking selection (see Section~\ref{sec:selection-methods},
page~\pageref{method:ranking-preselect}) to select one organism at a
time from the old (current) population, then based on a random draw applies either a
random respectful recombination \term{crossover} (see Section~\ref{method:r3-cross-organisms},
page~\pageref{method:r3-cross-organisms}) with another
member of the old population (selected randomly), a \inxmethod{swap-alleles} \term{mutate}
operator (see Section~\ref{method:swap-alleles}, page~\pageref{method:swap-alleles}),
or simple reproduction, to supply members of the new population.

\filbreak
\begin{clcode}(defvar *SSE-POP-SUBSET-SIZE* 0.6
  {\sf "Subset of the old population from which ranking selection will be drawn."})

(defmethod OPERATE-ON-POPULATION ((plan sse-plan) old-pop new-pop &AUX
                                  (new-orgs (organisms new-pop))
                                  (p-cross (prob-cross plan))
                                  (p-mutate (+ p-cross (prob-mutate plan)))
                                  (pop-size (size new-pop))
                                  (orphan (make-instance
                                           (organism-class old-pop))))
  (do* ((generator (ranking-preselect old-pop
                                      :multiplier *sse-pop-subset-size*)
                   (ranking-preselect old-pop
                                      :multiplier *sse-pop-subset-size*))
        (org1 (funcall generator)
              (funcall generator))
        org2
        (random# (geco-random-float 1.0) (geco-random-float 1.0))
        (i 0 (1+ i)))
       ((>= i pop-size))
    (cond
     ((> p-cross random#)
      (if (< (hamming-distance
              (first (genotype org1))
              (first (genotype (setf org2 (funcall generator))))))
          (r3-cross-organisms
           org1 org2
           (setf (aref new-orgs i) (make-organism new-pop))
           orphan ;a throw-away
           :allele-test #'=)
        ;; {\sf hamming distances less than 2 will produce eidetic offspring anyway,}
        ;; {\sf so bypass crossover}
        (setf (aref new-orgs i)
              (copy-organism-with-score org1 :new-population new-pop))))
     ((> p-mutate random#)
      (swap-alleles (setf (aref new-orgs i)
                          (copy-organism org1 :new-population new-pop))))
     (T ;; {\sf copying the score bypasses the need for a redundant evaluation}
      (setf (aref new-orgs i)
            (copy-organism-with-score org1 :new-population new-pop))))))\end{clcode}
\filbreak

The remaining code in \verb|ss-test.lisp| simply provides a test harness 
to repeatedly invoke the GAs, and accumulate performance information over 
a specified number of runs.

